\documentclass{beamer}
\usepackage[utf8]{inputenc}
\usepackage[T1]{fontenc}
\usepackage[french]{babel}
\usepackage{eurosym}

\begin{document}

%%%%%%%%%%%%%%
%%% diapo 1 %%%
%%%%%%%%%%%%%%
\begin{frame}
\frametitle{Jour 2}
\framesubtitle{Formation Obligatoire CHSCT}
\end{frame}

%%%%%%%%%%%%%%
%%% diapo 2 %%%
%%%%%%%%%%%%%%
\begin{frame}
\frametitle{Évaluation et prévention des risques professionnels}
\end{frame}

%%%%%%%%%%%%%%
%%% diapo 3 %%%
%%%%%%%%%%%%%%
\begin{frame}
\frametitle{Sommaire}
\begin{itemize}
\item Introduction
\item Enjeux juridiques de la prévention de la santé au travail
\item L’approche des risques professionnels : notions fondamentales
\item Les familles de risques professionnels
\item La préparation de la démarche d’évaluation des risques professionnels
\item La démarche d’évaluation des risques professionnels et les méthodes
\item La formalisation du document unique
\item L’hygiène générale
\end{itemize}
\end{frame} 

%%%%%%%%%%%%%%
%%% diapo 4 %%%
%%%%%%%%%%%%%%
\begin{frame}
\frametitle{Les évolutions du travail}
\framesubtitle{Evolutions tendancielles du travail ayant un impact sur la santé sécurité}
\end{frame} 

 %%%%%%%%%%%%%%
%%% diapo 5 %%%
%%%%%%%%%%%%%%
\begin{frame}
\frametitle{Exercice 1: Les évolutions du travail}
Comment à évolué le monde du travail ces 15/20 dernières années ?
\end{frame} 

%%%%%%%%%%%%%%
%%% diapo 6 %%%
%%%%%%%%%%%%%%
\begin{frame}
\frametitle{Des évolutions économiques}
\textbf{Des mutations profondes qui touchent toute l’économie}  
\begin{itemize}
\item Un modèle d’hyper-compétition (coût, qualité, délai, sécurité, innovation…) mondialisée.
\item Une financiarisation de l’économie : la performance financière l’emporte sur la dimension sociale.
\item Le diktat du client : accélération et augmentation des exigences des clients.
\item Développement des TIC ayant accentué la contraction du temps et de l’espace.
\item L’ère du changement permanent :
\begin{enumerate}
\item Fréquence accrue des réorganisations internes,
\item Externalisation, fusion, changement de statut… 
\end{enumerate}
\end{itemize}
\end{frame} 

%%%%%%%%%%%%%%
%%% diapo 7 %%%
%%%%%%%%%%%%%%
\begin{frame}
\frametitle{Des évolutions organisationnelles}
\textbf{Des conséquences en termes de nouveaux modes d’organisation, de nouvelles façon de travailler :}  
\begin{itemize}
\item Charge de travail en augmentation, diminution des temps de récupération.
\item Éclatement géographique du collectif de travail.
\item Diminution des temps de coopération formels ou informels.
\item Centralisation des organisations éloignant les salariés des centres de décision et  dépossédant le management de proximité de son pouvoir et de sa légitimité.
\item Individualisation des modes de management et de gestion des compétences.
\item Développement du reporting permanent.
\item Multiplication des normes, organisation de l’activité autour de process et d’outils très structurants, 
rationalisation des processus de production et des outils de gestion.
\item Développement de la polyvalence à outrance.
\end{itemize}
\end{frame} 

%%%%%%%%%%%%%%
%%% diapo 8 %%%
%%%%%%%%%%%%%%
\begin{frame}
\frametitle{Des évolutions sociétales}
\textbf{Des mutations sociologiques à prendre en compte}  
\begin{itemize}
\item Prévalence des pathologies mentales (dépression notamment).
\item Peur du chômage, incertitude sur l’avenir.
\item Atténuation de la frontière vie privée/vie professionnelle.
\item Flexibilité des horaires.
\item Vulgarisation de la psychologie(épanouissement individuel, bien-être, réussir sa vie,…).
\end{itemize}
\textbf{Un rapport au travail qui change}
\begin{itemize}
\item Les aspirations des générations - l’équilibre de vie au centre.
\item La recherche de sens, d’un travail « pour vivre » et plus d’une « vie de travail ».
\item Parfois des attentes en matière de lien social vis-à-vis des entreprises dans une société qui 
s’individualise.
\item Des exigences nouvelles vis-à-vis de son entreprise, et de son manager (RSE, éthique, valeurs 
partagées,…).
\end{itemize}
\end{frame} 

%%%%%%%%%%%%%%
%%% diapo 9 %%%
%%%%%%%%%%%%%%
\begin{frame}
\frametitle{Des évolutions sociales et humaines}
\textbf{Des contraintes accrues pesant sur les individus}
\begin{itemize}
\item Densification et intensification du travail.
\item Nécessité d’adaptation permanente au changement.
\item Affaiblissement des collectifs de travail, réduction des effectifs.
\item Fin des parcours tout tracés, instabilité/incertitudes sur l’emploi.
\item Allongement de la durée du travail, réduction du temps de travail.
\item Vieillissement de la population.
\item Affaiblissement de la valeur « qualité », qualité de travail empêchée: on recherche moins des experts 
ou spécialistes que des travailleurs « adaptables ».
\end{itemize}
\textbf{Avec une multiplication d’injonctions paradoxales}
\begin{itemize}
\item Isolement croissant des personnes vs interdépendance.
\item Faire preuves d’initiative, d’autonomie, mise en responsabilité, mais peu de marge de manœuvre (process 
rigides, etc.) dans un monde de process/d’outils normés.
\item Faible visibilité.
\item Faire vite et bien.
\end{itemize}
\end{frame}

%%%%%%%%%%%%%%
%%% diapo 10 %%%
%%%%%%%%%%%%%%
\begin{frame}
\frametitle{Enjeux juridiques de la prévention de la santé au travail}
\end{frame} 

%%%%%%%%%%%%%%
%%% diapo 11 %%%
%%%%%%%%%%%%%%
\begin{frame}
\frametitle{Repères historiques}
\textbf{schéma}
\end{frame} 

%%%%%%%%%%%%%%
%%% diapo 12 %%%
%%%%%%%%%%%%%%
\begin{frame}
\frametitle{Evolution récente de la réglementation sur la santé au travail de 1989 à 2014}
\textbf{schéma}
\end{frame}

%%%%%%%%%%%%%%
%%% diapo 13 %%%
%%%%%%%%%%%%%%
\begin{frame}
\frametitle{La Réglementation}
\framesubtitle{Contexte législatif et réglementaire pour l’évaluation des risques}
\textbf{schéma}
\end{frame}

%%%%%%%%%%%%%%
%%% diapo 14 %%%
%%%%%%%%%%%%%%
\begin{frame}
\frametitle{La Réglementation}
\framesubtitle{Contexte législatif et réglementaire pour l’évaluation des risques}
Directive Cadre n; 89/391/CEE du Conseil des Communautés Européennes du 12 juin 1989, (Directive Cadre) définit, en son article 6, les principes fondamentaux de la protection des travailleurs.
\textbf{EXIGENCES :}
\begin{itemize}
\item Obligation pour l’employeur d’assurer la santé et la sécurité des travailleurs.
\item Mise en œuvre des principes généraux de prévention des risques professionnels.
\item Obligation de procéder à l’évaluation, à priori, des risques.
\end{itemize}
\end{frame}

%%%%%%%%%%%%%%
%%% diapo 15 %%%
%%%%%%%%%%%%%%
\begin{frame}
\frametitle{La Réglementation}
\framesubtitle{Contexte législatif et réglementaire pour l’évaluation des risques}
La \textbf{loi n. 91-1414  du 31 décembre 1991} : la prévention des risques professionnels
\textbf{ARTICLE L.230-2 : OBLIGATION DE L’EMPLOYEUR}
\textbf{« Le chef d’établissement doit, compte tenu de la nature des activités de l’établissement, évaluer les risques pour la sécurité et la santé des travailleurs y compris dans le choix » :}
\begin{itemize}
\item des procédures de fabrication,
\item des équipements de travail,
\item des substances ou préparations chimiques,
\item des réaménagements des lieux de travail ou installations dans la définition des postes de travail ».
\end{itemize}
\end{frame}

%%%%%%%%%%%%%%
%%% diapo 16 %%%
%%%%%%%%%%%%%%
\begin{frame}
\frametitle{La Réglementation}
\framesubtitle{Le DUER}
\textbf{Le décret n° 2001-1016  du 5 novembre 2001 achève la transcription en droit français, des articles 9 et 10 de 
la directive-cadre européenne en ce qui concerne :}
\begin{itemize}
\item La traçabilité des résultats de l’évaluation des risques
\item La mise à disposition des résultats de l’évaluation des risques aux acteurs internes et externes
\end{itemize}
\textbf{Le Document Unique d’Evaluation des Risques (DUER)}
\end{frame}

%%%%%%%%%%%%%%
%%% diapo 17 %%%
%%%%%%%%%%%%%%
\begin{frame}
\frametitle{La Réglementation}
\framesubtitle{Le DUER}
\textbf{En application du Décret du 05 Novembre 2001 le DUER est tenu à disposition :}
\begin{itemize}
\item Des membres du CHSCT, ou des délégués du personne.
\item Des personnes soumises à risques en l’absence de CHSCT ou DP
\item Du médecin du travail.
\item De l’inspecteur ou du contrôleur du travail.
\item Des agents du service prévention de la CRAM.
\item Des représentants des organismes professionnels (OPPBTP, Médecin inspecteur du travail et de la main d’œuvre).
\end{itemize}
\end{frame}

%%%%%%%%%%%%%%
%%% diapo 18 %%%
%%%%%%%%%%%%%%
\begin{frame}
\frametitle{La Réglementation}
\framesubtitle{Le DUER}
\textbf{Les points clef du Décret sur le DUER :}
\begin{itemize}
\item Évaluation des risques identifiés.
\item Dans chaque unité de travail.
\item Transcrite sur un document unique.
\end{itemize}	
\textbf{Mise à jour :}
\begin{itemize}
\item Annuellement.
\item Lors d’aménagement importants
Lors du recueil d’une information supplémentaire.
\end{itemize}
\textbf{SANCTIONS :}
\begin{itemize}
\item \textbf{L’absence du document unique après le 07 novembre 2002 ou le fait de ne pas le mettre à jour est puni d’une amende.}
\item \textit{Contravention de 5ème classe = 1500 \euro{}/3000\euro{} en cas de récidive (art. 131-13 du Code Pénal).}
\end{itemize}
\end{frame}

%%%%%%%%%%%%%%
%%% diapo 19 %%%
%%%%%%%%%%%%%%
\begin{frame}
\frametitle{La législation}
Tout « chef d’établissement » est tenu, 
en vertu de l’\textbf{obligation générale de sécurité qui lui incombe} (\textit{Loi N. 91-1414 du 31 décembre 1991}), d’évaluer les risques éventuels et de prendre toutes les mesures nécessaires pour :
\textbf{CHEF D’ETABLISSEMENT}
\begin{itemize}
\item \textbf{« Assurer la sécurité et protéger la santé des travailleurs de l’établissement »}
\item \textbf{« Favoriser la prévention des risques professionnels »}
\end{itemize}
\textit{Ainsi dès 1991,} \textbf{cette loi}\textit{ a permis de transposer, pour l'essentiel, les dispositions que la directive cadre ajoutait au droit français.}
\end{frame}

%%%%%%%%%%%%%%
%%% diapo 20 %%%
%%%%%%%%%%%%%%
\begin{frame}
\frametitle{Les responsabilités de l’employeur}
\textbf{Article L.4121-1 :}
\textbf{Le chef d'établissement prend les mesures nécessaires pour assurer la sécurité et protéger la santé des travailleurs de l'établissement, y compris les travailleurs temporaires. Ces mesures comprennent des actions de prévention des risques professionnels, d'information et de formation ainsi que la mise en place d'une organisation et de moyens adaptés.} Il veille à l'adaptation de ces mesures pour \textbf{tenir compte du changement} des circonstances et tendre à l'amélioration des situations existantes. 
Sans préjudice des autres dispositions du présent code, lorsque dans un même lieu de travail les travailleurs de plusieurs entreprises sont présents, \textbf{les employeurs doivent coopérer à la mise en œuvre des dispositions relatives à la sécurité, à l'hygiène et à la santé} selon des conditions et des modalités définies par décret en Conseil d'Etat.
\textbf{La circulaire n.  6 DRT du 18 avril 2002 :}
\begin{itemize}
\item La jurisprudence qui s'établit impose à l'employeur une obligation de résultat en matière de sécurité et donne au manquement à cette obligation le caractère d'une faute inexcusable.
\item L'absence ou l'insuffisance du document unique établit automatiquement la faute inexcusable de l'employeur. 
\end{itemize}
\end{frame}

%%%%%%%%%%%%%%
%%% diapo 21 %%%
%%%%%%%%%%%%%%
\begin{frame}
\frametitle{Principes généraux de prévention}
\textbf{Le chef d'établissement met en œuvre les mesures sur la base des principes généraux de prévention suivants :}
\begin{itemize}
\item Eviter les risques.
\item Evaluer les risques qui ne peuvent pas être évités.
\item Combattre les risques à la source.
\item Adapter le travail à l’homme, en particulier en ce qui concerne la conception des postes de travail ainsi que  le choix des équipements et des méthodes de travail.
\item Tenir compte de l’état d’évolution de la technique.
\item Remplacer ce qui est dangereux par ce qui ne l’est pas ou ce qui l’est moins.
\item Planifier la prévention en y intégrant, dans un ensemble cohérent, la technique, l’organisation du travail, les conditions de travail, les relations sociales et l’influence des facteurs ambiants (y compris les risques liés au harcèlement moral).
\item Prendre des mesures de protection collective en leur donnant la priorité sur les mesures de protection individuelle.
\item Donner les instructions appropriées aux personnels.
\end{itemize}
\end{frame}

%%%%%%%%%%%%%%
%%% diapo 22 %%%
%%%%%%%%%%%%%%
\begin{frame}
\frametitle{L’approche des risques professionnels : notions fondamentales}
\end{frame} 

%%%%%%%%%%%%%%
%%% diapo 23 %%%
%%%%%%%%%%%%%%
\begin{frame}
\frametitle{Exercice 1: Les différentes notions}
Inscrivez sur une feuille toutes les idées qui vous viennent à l’esprit quand on vous dit : 
« risque professionnels », « dangers »
\end{frame} 

%%%%%%%%%%%%%%
%%% diapo 24 %%%
%%%%%%%%%%%%%%
\begin{frame}
\frametitle{Les risques professionnels}
\begin{itemize}
\item Liés aux \textbf{conditions de travail}, les \textbf{risques professionnels} font peser sur les salariés la menace d’une \textbf{altération de leur santé} qui peut se traduire par une \textbf{maladie} ou un \textbf{accident}.
\item \textbf{L’entreprise a l’obligation d'assurer la sécurité et protéger la santé de ses salariés face à tous types de risques professionnels. }
\item \textit{Tous les salariés sont concernés, qu'ils soient à temps plein ou partiel, temporaires, stagiaires, apprentis,…}
\end{itemize}
\end{frame} 

%%%%%%%%%%%%%%
%%% diapo 25 %%%
%%%%%%%%%%%%%%
\begin{frame}
\frametitle{La notion de risque et de danger}
\begin{itemize}
\item Le \textbf{danger} est une situation qui menace la sécurité et la santé des personnes, c’est une source potentielle de dommage, de préjudice ou d'effet nocif à l'égard d'une personne.
\begin{enumerate}
\item Le danger SE CONSTATE : c’est un état, une donnée préexistante
\end{enumerate}
\item Le \textbf{risque} est la situation de l’individu qui s’expose à un danger, c’est la probabilité qu'une personne subisse un préjudice ou des effets nocifs pour sa santé en cas d'exposition à un danger.
\begin{enumerate}
\item Le risque S’ÉVALUE : c’est une probabilité.
\end{enumerate}
\end{itemize}
\textbf{tableau}
\end{frame}

%%%%%%%%%%%%%%
%%% diapo 26 %%%
%%%%%%%%%%%%%%
\begin{frame}
\frametitle{Processus d’apparition d’un risque}
\textbf{schéma}
\end{frame}

%%%%%%%%%%%%%%
%%% diapo 27 %%%
%%%%%%%%%%%%%%
\begin{frame}
\frametitle{Processus d’apparition d’un risque}
\textbf{schéma}
\end{frame}


%%%%%%%%%%%%%%
%%% diapo 28 %%%
%%%%%%%%%%%%%%
\begin{frame}
\frametitle{Autres définitions}
\begin{itemize}
\item \textbf{Exposition} : Conditions dans lesquelles, le risque se présente à la victime
\item \textbf{Dommage} : Conséquence de la survenue de l’événement redouté. Assimilable au préjudice
\item \textbf{Vulnérabilité} : Faiblesse des protections face aux dangers
\item \textbf{Presque accident} : « Near miss » est un risque avéré. Il s’agit d’un événement concret et indiscutable sans qu’il y ait eu un accident.
\item \textbf{Accident du travail} : Accident survenu par le fait ou à l’occasion du travail
\item \textbf{Maladie professionnelle} : Conséquence de l’exposition plus ou moins prolongée, dans le cadre de l’activité professionnelle, à des agents physiques, chimiques ou infectieux (ayant entraîné une maladie).
\end{itemize}
\end{frame}

%%%%%%%%%%%%%%
%%% diapo 29 %%%
%%%%%%%%%%%%%%
\begin{frame}
\frametitle{Autres définitions}
\begin{itemize}
\item \textbf{Sécurité} : Etat dans lequel le risque de dommages corporels ou matériels est limité à un niveau acceptable.
\item \textbf{Santé} : Etat de complet bien-être physique, mental et social qui ne consiste pas seulement en une absence de maladie ou d’infirmité (OMS).
\item \textbf{Santé au travail} : Santé des employés, des travailleurs temporaires, et de toute autre personne présente sur le lieu de travail.
\end{itemize}
\end{frame}


%%%%%%%%%%%%%%
%%% diapo 30 %%%
%%%%%%%%%%%%%%
\begin{frame}
\frametitle{Pyramide des événements indésirables}
\framesubtitle{Pyramide de BIRD}
\textbf{schéma}
\end{frame}

%%%%%%%%%%%%%%
%%% diapo 31 %%%
%%%%%%%%%%%%%%
\begin{frame}
\frametitle{Les familles de risques professionnels}
\end{frame}

%%%%%%%%%%%%%%
%%% diapo 32 %%%
%%%%%%%%%%%%%%
\begin{frame}
\frametitle{Exercice 2: Les différents risques}
\textbf{Quels sont les risques professionnels que vous connaissez ?}
\end{frame}

%%%%%%%%%%%%%%
%%% diapo 33 %%%
%%%%%%%%%%%%%%
\begin{frame}
\frametitle{Typologie des risques}
\textbf{tableau}
\end{frame}

%%%%%%%%%%%%%%
%%% diapo 34 %%%
%%%%%%%%%%%%%%
\begin{frame}
\frametitle{Les familles de risques}
\textbf{Risque de chute de plain-pied}
\begin{itemize}
\item Risques de glissades, trébuchements, faux-pas et autres pertes d'équilibre sur une surface "plane". Les pertes d’équilibre peuvent être dues : à un mauvais état des sols, à la présence d’obstacles sur le sol…
\end{itemize}
\textbf{Risque de chute de hauteur}
\begin{itemize}
\item On dit qu'il y a risque de chute de hauteur lorsqu'il n'existe pas d'obstacle suffisamment efficace en bordure d'un vide. Les conséquences peuvent être très graves, d’autant plus lorsque le dénivelé est grand.
\end{itemize}
\end{frame}

%%%%%%%%%%%%%%
%%% diapo 35 %%%
%%%%%%%%%%%%%%
\begin{frame}
\frametitle{Les familles de risques}
\textbf{Risques liés aux circulations internes}
\begin{itemize}
\item Risques d’accident résultant du heurt d’une personne par un véhicule (motocyclette, voiture, camion, chariot de manutention,…) ou de la collision de véhicules entre eux ou contre un obstacle, au sein de l’entreprise. Ce sont des risques dont les conséquences peuvent être très graves, d’autant plus que l’énergie mise en œuvre est importante (vitesse, masse,…)
\end{itemize}
\textbf{Risque routier}
\begin{itemize}
\item C’est un risque d’accident de la circulation lié au déplacement d’un salarié réalisant une mission pour le compte de son entreprise. Ce risque présente des conséquences très importantes et est présent dans la quasi-totalité des entreprises.
\end{itemize}
\end{frame}

%%%%%%%%%%%%%%
%%% diapo 36 %%%
%%%%%%%%%%%%%%
\begin{frame}
\frametitle{Les familles de risques}
\textbf{Risques liés à l’activité physique}
\begin{itemize}
\item  Ce sont des risques d’accident et/ou de maladie professionnelle au niveau du tronc, des membres supérieurs et inférieurs consécutifs à des postures contraignantes, des efforts physiques intenses et/ou répétitifs, à des écrasements, à des chocs.
\end{itemize}
\textbf{Risques liés aux agents chimiques}
\begin{itemize}
\item Ce sont des risques d’infection, d’intoxication, d’allergie ou de brûlure par inhalation, ingestion ou contact cutané de produits mis en œuvre ou émis sous forme de gaz, de particules solides ou liquides.
\end{itemize}
\end{frame}

%%%%%%%%%%%%%%
%%% diapo 37 %%%
%%%%%%%%%%%%%%
\begin{frame}
\frametitle{Les familles de risques}
\textbf{Risques liés aux équipements de travail et aux engins mécaniques}
\begin{itemize}
\item Ce sont des risques d’accident causés par l’action mécanique (coupure, perforation, écrasement,...) d’une machine, d’une partie de machine, d’un outil portatif ou à main.
\end{itemize}
\textbf{Risques liés aux effondrements et aux chutes d’objets}
\begin{itemize}
\item Ce sont des risques d’accident qui résultent de la chute d’objets provenant de stockage, d’un étage supérieur ou de l’effondrement de matériaux.
\end{itemize}
\end{frame}


%%%%%%%%%%%%%%
%%% diapo 38 %%%
%%%%%%%%%%%%%%
\begin{frame}
\frametitle{Les familles de risques}
\textbf{Risques et nuisances liés aux bruits}
\begin{itemize}
\item  Risques d’accident générés par l’inconfort, l’entrave à la communication orale et la gêne lors de l’exécution de tâches délicates. Ce sont également des risques de maladie professionnelle dans le cas d’une longue exposition : par exemple la surdité.
\end{itemize}
\textbf{Risques liés aux ambiances thermiques}
\begin{itemize}
\item  Ce sont des risques d’atteinte à la santé (malaises, fatigue, inconfort) si les conditions thermiques sont inadaptées.
\end{itemize}
\textbf{Risques liés à l’éclairage}
\begin{itemize}
\item  Risques d’atteintes à la santé (fatigue et gêne) si l’éclairage ou la  luminosité est inadapté. Il est aussi un facteur relativement fréquent d’accident (chute, heurt,…) ou d’erreur.
\end{itemize}
\end{frame}

%%%%%%%%%%%%%%
%%% diapo 39 %%%
%%%%%%%%%%%%%%
\begin{frame}
\frametitle{Les familles de risques}
\textbf{Risques d’incendie ou d’explosion}
\begin{itemize}
\item Ce sont des risques d’accident (brûlure, blessure) consécutifs à un incendie ou à une explosion de produits inflammables.
\end{itemize}
\textbf{Risques liés aux agents biologiques}
\begin{itemize}
\item Risques d’infection, d’allergie ou d’intoxication résultant de la présence de micro organismes (bactéries, virus, moisissures,…). Le mode de transmission peut se faire par inhalation, par ingestion, par contact ou par pénétration suite à une lésion.
\end{itemize}
\end{frame}

%%%%%%%%%%%%%%
%%% diapo 40 %%%
%%%%%%%%%%%%%%
\begin{frame}
\frametitle{Les familles de risques}
\textbf{Risques liés à l’électricité}
\begin{itemize}
\item Ce sont des risques d’accident (brûlure ou électrocution) consécutifs à un contact avec un conducteur électrique ou une partie métallique sous tension.
\end{itemize}
\textbf{Risques liés aux rayonnements}
\begin{itemize}
\item Ce sont des risques d’accident et d’atteinte plus ou moins grave pour la santé selon le type et la puissance des rayonnements. Ils peuvent être émis par certains appareils ou provenir spontanément de matériaux.
\end{itemize}
\end{frame}

%%%%%%%%%%%%%%
%%% diapo 41 %%%
%%%%%%%%%%%%%%
\begin{frame}
\frametitle{Les familles de risques}
\textbf{Risques liés à la manutention mécanique}
\begin{itemize}
\item Risques d’accidents liés à l’utilisation d’appareils ou matériels de levage fixes ou mobiles
\end{itemize}
\textbf{Risques psychosociaux}
\begin{itemize}
\item Stress, mal être, souffrance au travail, burn-out, violence au travail, harcèlement moral au travail, harcèlement sexuel au travail, suicide en lien avec le travail
\end{itemize}
\end{frame}

%%%%%%%%%%%%%%
%%% diapo 42 %%%
%%%%%%%%%%%%%%
\begin{frame}
\frametitle{Exercice 3: Les différents risques}
\textbf{Cas pratique : déterminer si la situation de travail présentée constitue un risque ou pas et lequel.} 
\end{frame}

%%%%%%%%%%%%%%
%%% diapo 43 %%%
%%%%%%%%%%%%%%
\begin{frame}
\frametitle{Exercice 3: Les différents risques}
\textbf{photos}
\end{frame}

%%%%%%%%%%%%%%
%%% diapo 44 %%%
%%%%%%%%%%%%%%
\begin{frame}
\frametitle{Exercice 3: Les différents risques}
\textbf{photos}
\end{frame}

%%%%%%%%%%%%%%
%%% diapo 45 %%%
%%%%%%%%%%%%%%
\begin{frame}
\frametitle{Exercice 3: Les différents risques}
\textbf{photos}
\end{frame}

%%%%%%%%%%%%%%
%%% diapo 46 %%%
%%%%%%%%%%%%%%
\begin{frame}
\frametitle{Exercice 3: Les différents risques}
\textbf{photos}
\end{frame}

%%%%%%%%%%%%%%
%%% diapo 47 %%%
%%%%%%%%%%%%%%
\begin{frame}
\frametitle{Exercice 3: Les différents risques}
\textbf{photos}
\end{frame}

%%%%%%%%%%%%%%
%%% diapo 48 %%%
%%%%%%%%%%%%%%
\begin{frame}
\frametitle{Exercice 3: Les différents risques}
\textbf{photos}
\end{frame}

%%%%%%%%%%%%%%
%%% diapo 49 %%%
%%%%%%%%%%%%%%
\begin{frame}
\frametitle{Exercice 3: Les différents risques}
\textbf{photos}
\end{frame}

%%%%%%%%%%%%%%
%%% diapo 50 %%%
%%%%%%%%%%%%%%
\begin{frame}
\frametitle{Exercice 3: Les différents risques}
\textbf{photos}
\end{frame}

%%%%%%%%%%%%%%
%%% diapo 51 %%%
%%%%%%%%%%%%%%
\begin{frame}
\frametitle{Exercice 3: Les différents risques}
\textbf{photos}
\end{frame}

%%%%%%%%%%%%%%
%%% diapo 52 %%%
%%%%%%%%%%%%%%
\begin{frame}
\frametitle{Exercice 3: Les différents risques}
\textbf{photos}
\end{frame}

%%%%%%%%%%%%%%
%%% diapo 53 %%%
%%%%%%%%%%%%%%
\begin{frame}
\frametitle{Exercice 3: Les différents risques}
\textbf{photos}
\end{frame}

%%%%%%%%%%%%%%
%%% diapo 54 %%%
%%%%%%%%%%%%%%
\begin{frame}
\frametitle{Exercice 3: Les différents risques}
\textbf{photos}
\end{frame}

%%%%%%%%%%%%%%
%%% diapo 55 %%%
%%%%%%%%%%%%%%
\begin{frame}
\frametitle{Exercice 3: Les différents risques}
\textbf{photos}
\end{frame}

%%%%%%%%%%%%%%
%%% diapo 56 %%%
%%%%%%%%%%%%%%
\begin{frame}
\frametitle{Exercice 3: Les différents risques}
\textbf{photos}
\end{frame}

%%%%%%%%%%%%%%
%%% diapo 57 %%%
%%%%%%%%%%%%%%
\begin{frame}
\frametitle{Exercice 3: Les différents risques}
\textbf{photos}
\end{frame}

%%%%%%%%%%%%%%
%%% diapo 58 %%%
%%%%%%%%%%%%%%
\begin{frame}
\frametitle{Les risques psychosociaux}

\end{frame}

%%%%%%%%%%%%%%
%%% diapo 59 %%%
%%%%%%%%%%%%%%
\begin{frame}
\frametitle{Les Risques Psycho-Sociaux (RPS)}
\begin{itemize}
\item Les \textbf{Risques Psycho-Sociaux ou RPS} désignent l’ensemble des \textbf{facteurs individuels} (psycho), \textbf{organisationnels & relationnels} (sociaux) sur le lieu de travail qui peuvent avoir un \textbf{impact sur la santé physique et/ou mentale des individus} (Vézina, 2002).
\item Risques découlant de l’\textbf{interaction entre les individus et de l’interaction de l’individu avec son travail} (INRS)
\end{itemize}
\end{frame}

%%%%%%%%%%%%%%
%%% diapo 60 %%%
%%%%%%%%%%%%%%
\begin{frame}
\frametitle{Les risques psychosociaux/facteurs psychosociaux}
\textbf{Facteurs psychosociaux/risques psychosociaux :}
\begin{itemize}
\item Les facteurs psychosociaux
\begin{enumerate}
\item Les facteurs psychosociaux au travail désignent un vaste ensemble de variables, à l'intersection des dimensions individuelles, collectives et organisationnelles de l'activité professionnelle.
\end{enumerate}
\item \textbf{Les risques psychosociaux}
\begin{enumerate}
\item Les risques psychosociaux recouvrent des risques professionnels qui portent atteinte à l’intégrité physique et à la santé mentale des salariés: \textit{stress, harcèlement, épuisement professionnel, violence au travail...}
\end{enumerate}
\end{itemize}
\end{frame}

%%%%%%%%%%%%%%
%%% diapo 61 %%%
%%%%%%%%%%%%%%
\begin{frame}
\frametitle{Les différents risques psychosociaux}
\begin{itemize}
\item Stress chronique
\item Violences et incivilités
\item Harcèlements
\item Souffrance psychique
\item Troubles musculo-squelettiques
\item Conduites suicidaires
\item Troubles psycho-traumatiques
\item Conduites addictives
\item Épuisement professionnel (burnout)
\end{itemize}
\end{frame}

%%%%%%%%%%%%%%
%%% diapo 62 %%%
%%%%%%%%%%%%%%
\begin{frame}
\frametitle{Les facteurs psychosociaux}
\textbf{tableau}
\end{frame}


%%%%%%%%%%%%%%
%%% diapo 63 %%%
%%%%%%%%%%%%%%
\begin{frame}
\frametitle{Propriétés des facteurs psychosociaux}
Différentes études menées sur les facteurs de RPS au travail montrent qu’ils sont d’autant plus « toxiques » pour la santé quand :
\begin{itemize}
\item Ils s’inscrivent dans la durée
\item Ils sont subis
\item Ils sont nombreux
\item Ils sont incompatibles
\end{itemize}
\end{frame}

%%%%%%%%%%%%%%
%%% diapo 64 %%%
%%%%%%%%%%%%%%
\begin{frame}
\frametitle{Exo}
\textbf{Vidéo}
\end{frame}

%%%%%%%%%%%%%%
%%% diapo 65 %%%
%%%%%%%%%%%%%%
\begin{frame}
\frametitle{Exo}
Citer des dommages possibles correspondants à chacun des dangers énuméré ? 
\end{frame}

%%%%%%%%%%%%%%
%%% diapo 66 %%%
%%%%%%%%%%%%%%
\begin{frame}
\frametitle{Exo}
\textbf{Tableau}
\end{frame}

%%%%%%%%%%%%%%
%%% diapo 67 %%%
%%%%%%%%%%%%%%
\begin{frame}
\frametitle{Exo}
\textbf{Tableau}
\end{frame}

%%%%%%%%%%%%%%
%%% diapo 68 %%%
%%%%%%%%%%%%%%
\begin{frame}
\frametitle{Exo}
\textbf{Tableau}
\end{frame}

%%%%%%%%%%%%%%
%%% diapo 69 %%%
%%%%%%%%%%%%%%
\begin{frame}
\frametitle{Exo}
\textbf{Tableau}
\end{frame}

%%%%%%%%%%%%%%
%%% diapo 70 %%%
%%%%%%%%%%%%%%
\begin{frame}
\frametitle{Les indicateurs d’alerte}
\textbf{Tableau}
\end{frame}

%%%%%%%%%%%%%%
%%% diapo 71 %%%
%%%%%%%%%%%%%%
\begin{frame}
\frametitle{La préparation de la démarche d’évaluation des risques professionnels}
\end{frame}

%%%%%%%%%%%%%%
%%% diapo 72 %%%
%%%%%%%%%%%%%%
\begin{frame}
\frametitle{Prévenir c’est….}
\textbf{schéma}
\end{frame}

%%%%%%%%%%%%%%
%%% diapo 73 %%%
%%%%%%%%%%%%%%
\begin{frame}
\frametitle{Actions préalables}
\begin{itemize}
\item \textbf{CHOISIR les acteurs qui participent à l’évaluation des risques :} 
chef d’entreprise, CHSCT / DP, Ingénieur sécurité, salariés, médecin du travail,…
\item \textbf{LISTER}
\begin{enumerate}
\item  \textbf{les informations à recueillir} (organigramme, bilan social, fiche de poste…) 
\item  \textbf{les méthodes d’investigations à employer} (analyse documentaire, observations, entretiens, questionnaire…)
\end{enumerate}
\item \textbf{DECOUPER} l’organisation en unités de travail.
\end{itemize}
\end{frame}

%%%%%%%%%%%%%%
%%% diapo 74 %%%
%%%%%%%%%%%%%%
\begin{frame}
\frametitle{Découpage de l’organisation en unités de travail}
\textbf{Définition : }
Les \textbf{situations de travail identiques pour lesquelles les risques et les dangers sont similaires} sont décomposées en \textbf{unités de travail} (UT) et éventuellement en sous unités de travail (SUT).
Plusieurs façons de définir les unités de travail et les sous unités de travail sont possibles :
\begin{itemize}
\item \textit{Ex. 1 =les UT en fonction des lieux de travail ou métiers}
\item \textit{Ex. 2 = les UT en fonction des lieux de travail et les SUT en  fonction des métiers.}
\end{itemize}
\end{frame}

%%%%%%%%%%%%%%
%%% diapo 75 %%%
%%%%%%%%%%%%%%
\begin{frame}
\frametitle{Cas pratique}
\textbf{Découper une association d’aide à domicile en UT et SUT}
\textit{Réponse : }
On peut par exemple dénombrer deux unités de travail :
\begin{itemize}
\item UT 1 : le domicile de la personne aidée.
\item UT 2 : les locaux de l’association.
\end{itemize}
Et on peut dénombrer trois sous unité de travail pour UT 1 :
\begin{itemize}
\item SUT1 : les infirmiers
\item SUT1 bis : les aides-soignants
\item SUT1 ter : les aides à domicile
\end{itemize}
\end{frame}

%%%%%%%%%%%%%%
%%% diapo 76 %%%
%%%%%%%%%%%%%%
\begin{frame}
\frametitle{Exemple 1: Tableau d’évaluation des risques professionnels}
\textbf{tableau}
\end{frame}

%%%%%%%%%%%%%%
%%% diapo 77 %%%
%%%%%%%%%%%%%%
\begin{frame}
\frametitle{La démarche d’évaluation des risques professionnels et les méthodes}
\end{frame}

%%%%%%%%%%%%%%
%%% diapo 78 %%%
%%%%%%%%%%%%%%
\begin{frame}
\frametitle{La démarche d’évaluation des risques}
\framesubtitle{Analyser/évaluer les risques}
\textbf{schéma}
\end{frame}

%%%%%%%%%%%%%%
%%% diapo 79 %%%
%%%%%%%%%%%%%%
\begin{frame}
\frametitle{La démarche d’évaluation des risques}
\framesubtitle{Analyser/évaluer les risques}
\textbf{schéma}
\end{frame}

%%%%%%%%%%%%%%
%%% diapo 80 %%%
%%%%%%%%%%%%%%
\begin{frame}
\frametitle{La démarche d’évaluation des risques}
\framesubtitle{Analyser/évaluer les risques}
\textbf{schéma}
\end{frame}

%%%%%%%%%%%%%%
%%% diapo 81 %%%
%%%%%%%%%%%%%%
\begin{frame}
\frametitle{Identifier les risques professionnels}
\textbf{Répertorier les situations dangereuses par activité et identifier les risques associés }
Ex : Unité de travail : Jardinier
Pour l’activité « Tailler » :
\textbf{Source de danger 1} :
Utilisation de dispositifs mobiles : échelle, et échafaudage pour tailler les haies
\textit{Risque 1} : Risque de chute de hauteur
\textbf{Source de danger 2} : 
Utilisation d’outil motorisé : taille-haie
\textit{Risque 2} : Risque lié aux équipements de travail et engins mécaniques  
\end{frame}

%%%%%%%%%%%%%%
%%% diapo 82 %%%
%%%%%%%%%%%%%%
\begin{frame}
\frametitle{Identifier les risques professionnels}
\textbf{tableau}
\end{frame}

%%%%%%%%%%%%%%
%%% diapo 83 %%%
%%%%%%%%%%%%%%
\begin{frame}
\frametitle{Identifier les risques professionnels}
\textbf{tableau}
\end{frame}

%%%%%%%%%%%%%%
%%% diapo 84 %%%
%%%%%%%%%%%%%%
\begin{frame}
\frametitle{Cotation des risques}
Pour les coter il faut prendre en compte 2 critères (méthode de classification de l’INRS) :
\begin{itemize}
\item \textbf{le niveau de gravité potentielle} lié à la nature de la lésion ou de l’atteinte à la santé susceptible d’être provoquée (G)
\item \textbf{le niveau d’exposition} lié à la fréquence d’exposition au danger
\begin{enumerate}
\item \textbf{Le produit (G) x (F) détermine le niveau de risque.}
\item \textbf{Les différents seuils de niveaux de risque permettent de définir les niveaux de priorité des actions à mener pour réduire le risque.}
\end{enumerate}
\end{itemize}
\end{frame}

%%%%%%%%%%%%%%
%%% diapo 85 %%%
%%%%%%%%%%%%%%
\begin{frame}
\frametitle{Classer les risques}
\textbf{tableau}
\end{frame}

%%%%%%%%%%%%%%
%%% diapo 86 %%%
%%%%%%%%%%%%%%
\begin{frame}
\frametitle{Classer les risques}
\textbf{tableau}
\end{frame}

%%%%%%%%%%%%%%
%%% diapo 87 %%%
%%%%%%%%%%%%%%
\begin{frame}
\frametitle{Classer les risques}
\textbf{tableau}
\end{frame}

%%%%%%%%%%%%%%
%%% diapo 88 %%%
%%%%%%%%%%%%%%
\begin{frame}
\frametitle{La démarche d’évaluation des risques}
\framesubtitle{Les différents niveaux de prévention}
\textbf{tableau}
\end{frame}

%%%%%%%%%%%%%%
%%% diapo 89 %%%
%%%%%%%%%%%%%%
\begin{frame}
\frametitle{Le Document Unique d’Evaluation des Risques}
\framesubtitle{DUER}
\end{frame}

%%%%%%%%%%%%%%
%%% diapo 90 %%%
%%%%%%%%%%%%%%
\begin{frame}
\frametitle{Rappel: Qu’est-ce qu’un document unique ?}
\textbf{DUER}
\begin{itemize}
\item Transcription écrite de l’évaluation des risques professionnels
\item Recensement des risques pour la santé et la sécurité des travailleurs
\end{itemize}
\end{frame}

%%%%%%%%%%%%%%
%%% diapo 91 %%%
%%%%%%%%%%%%%%
\begin{frame}
\frametitle{Le Document Unique}
\textbf{La tenue d’un Document Unique d’Évaluation des Risques (DUER) par l’employeur est devenue obligatoire depuis 2001.}
\begin{itemize}
\item Il est le bilan de l’analyse de l’ensemble des risques de l’établissement. 
\item Il doit être révisé au moins une fois par an et à chaque changement important pouvant influencer sur la sécurité et les conditions de travail.
\end{itemize}
\textbf{Il doit contenir et a pour objectif :}
\begin{itemize}
\item une identification des risques
\item une évaluation des risques
\item le classement des risques
\item des propositions d’action de prévention.
\end{itemize}
Ces étapes permettent la mise en place d’un plan d’action.
\end{frame}

%%%%%%%%%%%%%%
%%% diapo 92 %%%
%%%%%%%%%%%%%%
\begin{frame}
\frametitle{La forme du DUER}
\begin{itemize}
\item La réglementation ne prévoit aucun document « type » mais…
\item Il doit nécessairement prendre la forme d’un support unique qui peut être aussi bien un document papier que numérique. 
\item Si le support numérique comporte des informations nominatives, une déclaration préalable doit être effectuée auprès de la Commission nationale de l’informatique et des libertés (CNIL). 
\end{itemize}
\end{frame}

%%%%%%%%%%%%%%
%%% diapo 93 %%%
%%%%%%%%%%%%%%
\begin{frame}
\frametitle{Le contenu du DU}
\begin{itemize}
\item Comme son nom l’indique, le document unique doit rassembler en un seul document toutes les données concernant les dangers et les risques professionnels identifiés.
\item Le document unique doit comporter un inventaire des risques identifiés dans chaque unité de travail de l’entreprise ou de l’établissement.  
\end{itemize}
\end{frame}

%%%%%%%%%%%%%%
%%% diapo 94 %%%
%%%%%%%%%%%%%%
\begin{frame}
\frametitle{Le DUER}
\textbf{schéma}
\end{frame}

%%%%%%%%%%%%%%
%%% diapo 95 %%%
%%%%%%%%%%%%%%
\begin{frame}
\frametitle{Cotation du DUER}
\textbf{schéma}
\end{frame}

%%%%%%%%%%%%%%
%%% diapo 96 %%%
%%%%%%%%%%%%%%
\begin{frame}
\frametitle{Mise en œuvre des actions}
\textbf{Élaborer un programme d’actions }
Les actions peuvent consister en :
\begin{itemize}
\item Des travaux liés aux équipements de travail
\item Un aménagement des locaux
\item Des actions de formation, d’information
\end{itemize}
\textbf{Mise à jour du DU}
\end{frame}

%%%%%%%%%%%%%%
%%% diapo 97 %%%
%%%%%%%%%%%%%%
\begin{frame}
\frametitle{Mise à jour}
\textbf{Au moins annuelle}
\begin{itemize}
\item En cas d’information supplémentaire
\item Suite aux actions 
\item Des changements techniques ou des changements organisationnels sont susceptibles de générer de nouveaux risques, il faudra donc effectuer une nouvelle évaluation des risques.
\end{itemize}
\textbf{Il est conseillé de charger une personne du suivi du document unique (ex. : chef de service / représentant, responsable RH, responsable qualité, responsable des moyens généraux…).
La désignation de cette personne n’exonère pas l’employeur de sa responsabilité.}
\end{frame}

%%%%%%%%%%%%%%
%%% diapo 98 %%%
%%%%%%%%%%%%%%
\begin{frame}
\frametitle{Accessibilité du document}
Il doit être tenu à la disposition :
\begin{itemize}
\item des délégués du personnel et du comité d’hygiène, de sécurité et des conditions de travail (CHSCT) ou, à défaut, des agents ;
\item du médecin de prévention ;
\item des ISS et ACFI ;
\item des assistants et conseillers prévention.
\end{itemize}
Un exemplaire « papier » comportant la dernière date de mise à jour doit être conservé dans les locaux de la structure.
\end{frame}

%%%%%%%%%%%%%%
%%% diapo 99 %%%
%%%%%%%%%%%%%%
\begin{frame}
\frametitle{Cas pratique}
Exercice sur le document unique de l’entreprise → analyse des points forts et points faibles
\textit{Cas pratique : élaborer un modèle de document unique} 
\textbf{Cf. INRS – ED887}
\end{frame}

%%%%%%%%%%%%%%
%%% diapo 100 %%%
%%%%%%%%%%%%%%
\begin{frame}
\frametitle{L’hygiène générale}
\end{frame}

%%%%%%%%%%%%%%
%%% diapo 101 %%%
%%%%%%%%%%%%%%
\begin{frame}
\frametitle{Les vestiaires}
\textbf{Ils doivent être pourvus de sièges, d’armoires individuelles ininflammables.}
\end{frame}

%%%%%%%%%%%%%%
%%% diapo 102 %%%
%%%%%%%%%%%%%%
\begin{frame}
\frametitle{Les toilettes}
\textbf{Ils doivent être pourvus de sièges, d’armoires individuelles ininflammables.}
Il doit y avoir :
\begin{itemize}
\item cabinet et 1 urinoir pour 20 hommes 
\item 2 cabinets pour 20 femmes.
\end{itemize}
\end{frame}

%%%%%%%%%%%%%%
%%% diapo 103 %%%
%%%%%%%%%%%%%%
\begin{frame}
\frametitle{Les douches}
\textbf{Des  douches journalières doivent être mises à la disposition du personnel effectuant des travaux salissants.}
\end{frame}

%%%%%%%%%%%%%%
%%% diapo 104 %%%
%%%%%%%%%%%%%%
\begin{frame}
\frametitle{L’interdiction de fumer}
Il est interdit de fumer dans les locaux clos et couverts affectés à l’ensemble du personnel.
L’employeur établit un plan d’aménagement destiné à assurer la protection des non-fumeurs (actualisé tous les deux ans.)
\end{frame}

%%%%%%%%%%%%%%
%%% diapo 105 %%%
%%%%%%%%%%%%%%
\begin{frame}
\frametitle{En cas de danger grave et imminent}
Les agents/fonctionnaires bénéficient d’un droit d’alerte et de retrait s’ils ont un motif raisonnable de croire qu’une situation de travail présente un danger grave et imminent pour leur vie et leur santé.
\end{frame}

%%%%%%%%%%%%%%
%%% diapo 106 %%%
%%%%%%%%%%%%%%
\begin{frame}
\frametitle{La protection dans l’administration}
L’administration publique doit veiller :
\begin{itemize}
\item au confort du personnel,
\itemà sa sécurité au sein de l’établissement,
\item et doit éviter toute altération à la santé des travailleurs.
\end{itemize}
\end{frame}

%%%%%%%%%%%%%%
%%% diapo 107 %%%
%%%%%%%%%%%%%%
\begin{frame}
\frametitle{L’Intensité des bruits}
Elle doit être maintenue à un niveau compatible avec la santé des travailleurs.
Un contrôle doit être assuré et des dispositions prises lorsque le niveau sonore atteint plus de 80 à 85 décibels.
\end{frame}

%%%%%%%%%%%%%%
%%% diapo 108 %%%
%%%%%%%%%%%%%%
\begin{frame}
\frametitle{L’aération des locaux}
\textbf{schéma}
\end{frame}

%%%%%%%%%%%%%%
%%% diapo 109 %%%
%%%%%%%%%%%%%%
\begin{frame}
\frametitle{L’éclairage des locaux}
\textbf{schéma}
\end{frame}

%%%%%%%%%%%%%%
%%% diapo 110 %%%
%%%%%%%%%%%%%%
\begin{frame}
\frametitle{La restauration}
L’institution de tickets de restaurant en faveur du personnel peut pallier à l’installation d’un réfectoire.
\end{frame}

%%%%%%%%%%%%%%
%%% diapo 111 %%%
%%%%%%%%%%%%%%
\begin{frame}
\frametitle{Les produits dangereux}
Les  Fiches de données de sécurité (F.D.S) doivent être maintenues à jour.
\end{frame}

%%%%%%%%%%%%%%
%%% diapo 112 %%%
%%%%%%%%%%%%%%
\begin{frame}
\frametitle{La température}
L’employeur doit assurer la protection contre le froid et les intempéries.
\end{frame}

%%%%%%%%%%%%%%
%%% diapo 113 %%%
%%%%%%%%%%%%%%
\begin{frame}
\frametitle{L’eau potable}
L’employeur doit assurer la fourniture d’eau potable fraîche.
\end{frame}

\end{document}
