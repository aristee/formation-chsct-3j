\documentclass{beamer}
\usepackage[utf8]{inputenc}
\usepackage[T1]{fontenc}
\usepackage{enumitem}
\uselanguage{French}
\languagepath{French}
\setbeamertemplate{itemize item}[square]
\setbeamertemplate{itemize subitem}[triangle]
\setbeamertemplate{itemize subsubitem}[circle]

\begin{document}

%%%%%%%%%%%%%%
%%% diapo 1 %%%
%%%%%%%%%%%%%%

\begin{frame}
\frametitle{Jour 1}
\framesubtitle{FORMATION DES MEMBRES DU CHSCT}
\end{frame}

%%%%%%%%%%%%%%
%%% diapo 2 %%%
%%%%%%%%%%%%%%

\begin{frame}
\frametitle{Présentation}
juste une modification
	\begin{itemize}
		\item Présentation de la formation
		\item Présentation des participants
        \item Leur travail, leur poste / leur fonction, …
        \item Leurs attentes, leurs besoins, …
	\end{itemize}
\end{frame}

%%%%%%%%%%%%%%
%%% diapo 3 %%%
%%%%%%%%%%%%%%

\begin{frame}
\frametitle{Les objectifs de le formation}


- Maîtriser les règles de fonctionnement du CHSCT

- Identifier les moyens dont il dispose

- Clarifier les missions de prévention qui résultent des dispositions légales en vigueur

- Savoir détecter les risques professionnels

- Contribuer à l’amélioration des conditions de travail

- Réaliser efficacement les missions d’inspection et d’enquête

- Proposer des actions concrètes pour contribuer à la prévention

\end{frame}  

%%%%%%%%%%%%%%
%%% diapo 4 %%%
%%%%%%%%%%%%%%

\begin{frame}
\frametitle{Introduction}
\end{frame} 

%%%%%%%%%%%%%%
%%% diapo 5 %%%
%%%%%%%%%%%%%%

\begin{frame}
\frametitle{Historique des CHSCT}
- 1941 : Création des premiers Comités de Sécurité 

- 1947 : Ils deviennent des Comités d Hygiène et de Sécurité (CHS)

- 1973 : Création des Commissions pour l'Amélioration des Conditions  de Travail (CACT dans les entreprises de + 300 salariés)

- 1982 : Fusion des CHS et des CACT en Comité d'Hygiène, de Sécurité et des Conditions de Travail (CHSCT) 

- 1993 : Décret n93X449 du 23 mars 1993 précisant les dispositions concernant les CHSCT, complété
\end{frame} 

%%%%%%%%%%%%%%
%%% diapo 6 %%%
%%%%%%%%%%%%%%
\begin{frame}
\frametitle{Acteurs de la prévention : national}


- La CRAM/CARSAT = Caisse d'assurance retraite et de la santé au travail 

- Les organismes sous tutelle du ministère chargé du travail : 
\begin{itemize}[label=$\square$]
        \item ANACT = agence nationale pour l’amélioration des conditions de travail,
        \item OPPBTP = organisme professionnel de prévention du bâtiment et des travaux publics
\end{itemize}

- L’InVS = institut de veille sanitaire

- L’INRS 

- Autres : Observatoires nationaux,…

\end{frame} 

\end{document}


