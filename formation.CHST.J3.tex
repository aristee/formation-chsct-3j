\documentclass{beamer}
\usepackage[utf8]{inputenc}
\usepackage[T1]{fontenc}
\usepackage[french]{babel}
\usepackage{eurosym}

\begin{document}

%%%%%%%%%%%%%%
%%% diapo 1 %%%
%%%%%%%%%%%%%%
\begin{frame}
\frametitle{Jour 3}
\framesubtitle{Formation des membres du CHSCT}
\end{frame}

%%%%%%%%%%%%%%
%%% diapo 2 %%%
%%%%%%%%%%%%%%
\begin{frame}
\frametitle{Formation Obligatoire CHSCT}

\end{frame}

%%%%%%%%%%%%%%
%%% diapo 3 %%%
%%%%%%%%%%%%%%

\begin{frame}
\frametitle{Sommaire}

\begin{itemize}
\item \textbf{Maladie professionnelle et à caractère professionnelle}
\begin{enumerate}
\item Introduction
\item Définitions
\item 2 types de maladies
\item Réparation 
\item Prise en charge de la maladie suivant les cas
\end{enumerate}
\item \textbf{Accident du travail}
\begin{enumerate}
\item Définitions
\item Les statistiques d’accidents du travail
\item Les conséquences de l'accident 
\item Le mécanisme de l’accident du travail
\item La méthode de l’arbre des causes
\item Le recueil des faits
\item La construction de l’arbre des causes
\item Exploitation de l’arbre des causes et choix des mesures de prévention
\item Suivre l’application des solutions
\end{enumerate}
\end{itemize}
\end{frame} 

%%% diapo 4 %%%
%%%%%%%%%%%%%%

\begin{frame}
\frametitle{Maladie Professionnelle}

\end{frame} 

 %%%%%%%%%%%%%%
%%% diapo 5 %%%
%%%%%%%%%%%%%%

\begin{frame}
\frametitle{Maladie Professionnelle}
\framesubtitle{Définition}

\textbf{Pas de définition légale}

Contrairement à l'accident du travail et l'accident de trajet, il n'existe pas de définition légale générale de la maladie professionnelle. 

On peut toutefois indiquer qu'elle est la \textbf{conséquence de l'exposition plus ou moins prolongée} d’un travailleur à un \textbf{risque physique, chimique, biologique, ou résulte des conditions dans lesquelles il exerce son activité professionnelle}.
\end{frame} 



%%%%%%%%%%%%%%
%%% diapo 6 %%%
%%%%%%%%%%%%%%
\begin{frame}
\frametitle{Maladie Professionnelle}
\framesubtitle{Définition}

\textbf{3 conditions doivent être réunies pour permettre la prise en charge de la MP :}
\begin{itemize}
\item  Inscription de la maladie sur l’un des tableaux existants 
\item  Exposition au risque et en apporter les éléments de preuve. La liste des travaux est fixée par le tableau et peut être indicative ou limitative.
\item  Constatation de la maladie par un médecin dans un certain délai prévu par les tableaux et débutant à la fin de l'exposition au risque.
\end{itemize}
\end{frame} 

%%%%%%%%%%%%%%
%%% diapo 7 %%%
%%%%%%%%%%%%%%
\begin{frame}
\frametitle{Reconnaissance comme Maladie Professionnelle}

\textbf{Si un salarié :}
\begin{itemize}
\item tombe malade et que sa maladie semble être liée à son travail, 
\item est en arrêt de travail pour une affection liée à votre travail
\begin{enumerate}
\item \textbf{Celle-ci peut être reconnue d'origine professionnelle.} 
\item \textbf{Le salarié doit alors faire une demande de reconnaissance de la MP auprès de sa CPAM (formulaire CERFA n.60-3950). Au terme de l'instruction, la CPAM l’informe de sa décision.
Cerfa n.3950 Demande Reconnaissance MP.pdf}
\end{enumerate}

\item \textbf{Information de l'employeur}
\item La CPAM adresse à l’employeur une copie de votre déclaration de maladie professionnelle. L'employeur peut émettre des réserves motivées sur le caractère professionnel de la maladie
\end{itemize}
\end{frame} 

%%%%%%%%%%%%%%
%%% diapo 8 %%%
%%%%%%%%%%%%%%
\begin{frame}
\frametitle{Délai}

\begin{itemize}
\item Déclaration de la MP à la CPAM dans les \textbf{15 jours} suivant la cessation du travail. 
\begin{enumerate}
\item Si la maladie a été constatée avant son inscription au tableau des MP, le salarié peut alors déclarer sa maladie dans les \textbf{3 mois} suivant son inscription au tableau.
\end{enumerate}
\item \textbf{Si ces délais ne sont pas respectés, la déclaration reste recevable si elle est effectuée dans les 2 ans suivant :}
\begin{enumerate}
\item la date de l'arrêt du travail lié à la maladie, si le salarié a été informé avant l'arrêt de travail par certificat médical du lien possible entre sa maladie et son activité professionnelle,
\item ou la date à laquelle le salarié a été informé par certificat médical du lien possible entre sa maladie et son activité professionnelle (lorsque cette information est postérieure à sa cessation du travail),
\item ou la date de cessation du paiement des indemnités pour maladie,
\item ou la date de l’inscription de cette maladie aux tableaux des maladies professionnelles.
\end{enumerate}
\item La CPAM accuse réception de la déclaration de maladie professionnelle.
\end{itemize}
\end{frame} 


%%%%%%%%%%%%%%
%%% diapo 9 %%%
%%%%%%%%%%%%%%
\begin{frame}
\frametitle{Maladie Professionnelle}
\framesubtitle{Tableaux des maladies professionnelles}

\textbf{Les maladies professionnelles sont définies par des tableaux annexés à l’article R 461-3 du Code de la Sécurité Sociale (au nombre de 110)}

\textbf{Chaque tableau comporte 3 colonnes :}

\begin{itemize}
\item \textit{colonne 1} : les symptômes ou lésions pathologiques, énumérés de façon limitative, que  doit  présenter  le  malade

\item \textit{colonne 2} : le délai de prise en charge : délai maximal entre l’apparition de l’affection (date de la 1ere constatation) et  la  date  à  laquelle  le  travailleur  a  cessé d’être exposé au risque, (varie de 3 jours a 50 ans)

\item \textit{colonne 3} : liste limitative des travaux susceptibles de provoquer l’affection décrite en 1. 
\end{itemize}
\textit{Lien tableaux MP : http://www.inrs-mp.fr/mp/cgi-bin/mppage.pl?state=1&acc=5&gs=&rgm=2}
\end{frame}


%%%%%%%%%%%%%%
%%% diapo 10 %%%
%%%%%%%%%%%%%%
\begin{frame}
\frametitle{Maladie Professionnelle}
\framesubtitle{Principales Maladies Professionnelles - Statistiques}

\textbf{schéma}
\end{frame} 


%%%%%%%%%%%%%%
%%% diapo 11 %%%
%%%%%%%%%%%%%%
\begin{frame}
\frametitle{Maladie Professionnelle}
\framesubtitle{Points importants}

« La maladie professionnelle doit, quant à elle, être liée par une \textbf{relation de cause à effet avec le travail} pour être prise en charge »

« Elle est \textbf{reconnue par référence aux tableaux} des affections professionnelles »

« Ceux-ci ne sont pas limitatifs, il est \textbf{possible de reconnaître} un caractère professionnel à une affection non répertoriée »

\textbf{Preuve de la relation travail / maladie à apporter par le salarié}
\end{frame} 

%%%%%%%%%%%%%%
%%% diapo 12 %%%
%%%%%%%%%%%%%%
\begin{frame}
\frametitle{L’analyse d’un accident du travail}

\end{frame}


%%%%%%%%%%%%%%
%%% diapo 13 %%%
%%%%%%%%%%%%%%
\begin{frame}
\frametitle{Définitions}
\framesubtitle{L’analyse d'un AT}


\end{frame}

%%%%%%%%%%%%%%
%%% diapo 14 %%%
%%%%%%%%%%%%%%
\begin{frame}
\frametitle{Accidents du Travail}
\framesubtitle{Définition}

\textbf{Selon l’article L.411-1 du code de la sécurité sociale : }
\begin{itemize}
\item est considéré comme accident du travail, \textbf{quelle qu'en soit la cause}, l'accident survenu par le fait ou à l'occasion du travail à toute personne salariée ou travaillant, à quelque titre ou en quelque lieu que ce soit, pour un ou plusieurs employeurs ou chefs d'entreprise.

\item l’accident est caractérisé par l’\textbf{action violente et soudaine} d’une \textbf{cause extérieure} provoquant au \textbf{cours du travail} une \textbf{lésion} de l’organisme humain ou la \textbf{mort}.
\end{itemize}
\end{frame}

%%%%%%%%%%%%%%
%%% diapo 15 %%%
%%%%%%%%%%%%%%
\begin{frame}
\frametitle{Définition de l'accident de trajet}

\begin{itemize}
    \item \textbf{Est considéré comme accident de trajet, l’accident survenu au agent pendant le trajet aller et retour :}
\begin{enumerate}
\item \textbf{Entre son lieu de travail et sa résidence principale ou une résidence secondaire présentant un caractère de stabilité, ou tout autre lieu où le agent se rend de façon habituelle pour des motifs d’ordre familial ;}
\item \textbf{Entre son lieu de travail et le restaurant, la cantine, ou tout autre lieu où le agent prend habituellement ses repas.}
\end{enumerate}
\item Enfin, il doit s’agir d’un \textbf{itinéraire « normal »} c’est-à-dire du trajet \textbf{le plus direct}. 
\item En principe, le trajet ne doit pas être interrompu ou détourné. Le détour et l'interruption seront admis s'ils sont justifiés par l'emploi, par le covoiturage et par des nécessités de la vie courante telles que l'achat de nourriture ou des soins médicaux.
\end{itemize}
\end{frame}

%%%%%%%%%%%%%%
%%% diapo 16 %%%
%%%%%%%%%%%%%%
\begin{frame}
\frametitle{Accidents du Trajet}
\framesubtitle{Définition}

\textbf{Selon l’article L.411-1 du code de la sécurité sociale :}

\begin{itemize}
    \item est considéré comme accident de travail l’accident survenu pendant le trajet d’aller et retour entre : 
\begin{enumerate}
\item la résidence principale, une résidence secondaire présentant un caractère de stabilité ou tout autre lieu où le salarié se rend de façon habituelle pour des motifs d’ordre familial et le lieu de travail,
\item le lieu du travail et le lieu où le travailleur prend habituellement ses repas
\end{enumerate}
\item dans la mesure où le parcours n’a pas été interrompu ou détourné pour un motif dicté par l’intérêt personnel et étranger aux nécessités de la vie courante ou indépendant de l’emploi. 
\end{itemize}
\end{frame}



%%%%%%%%%%%%%%
%%% diapo 17 %%%
%%%%%%%%%%%%%%
\begin{frame}
\frametitle{Accidents du Trajet}
\framesubtitle{Périmètre}

\textbf{schéma}
\end{frame}

%%%%%%%%%%%%%%
%%% diapo 18 %%%
%%%%%%%%%%%%%%
\begin{frame}
\frametitle{Accidents du Trajet}
\framesubtitle{Analyse de Cas – 1}

\textbf{Les faits}

Une salariée fait une chute dans une station-service où elle s’était arrêtée en voiture avec son mari en rentrant de son travail.


\textbf{Accident de trajet ? }

\textbf{OUI} (Cass 1990)

\textbf{Les fondements du jugement}

le ravitaillement en essence s’intégrait dans le trajet et ne nécessitait pas de détour.
\end{frame}

%%%%%%%%%%%%%%
%%% diapo 19 %%%
%%%%%%%%%%%%%%
\begin{frame}
\frametitle{Accidents du Trajet}
\framesubtitle{Analyse de Cas – 2}

\textbf{Les faits}

Une salariée fait une chute dans son jardin au moment où elle allait prendre sa  voiture pour se rendre à 50 m de son travail. 

\textbf{Accident de trajet ? }

\textbf{NON} (Cass 1991)

\textbf{Les fondements du jugement}

L’accident est survenu dans un lieu où seule la victime est habilitée à prendre des mesures de prévention.

\end{frame}

%%%%%%%%%%%%%%
%%% diapo 20 %%%
%%%%%%%%%%%%%%
\begin{frame}
\frametitle{Accidents du Trajet}
\framesubtitle{Analyse de Cas – 3}

\textbf{Les faits}

Un salarié est victime d’un accident de circulation alors que, allant à son travail, il a fait un détour important pour acheter son pain. 



\textbf{Accident de trajet ? }

\textbf{OUI } (Cours d’appel 1992)

\textbf{Les fondements du jugement}

Le salarié subissait une contrainte de la vie courante d’autant que l’employeur n’offrait aucune  possibilité de restauration.
\end{frame}

%%%%%%%%%%%%%%
%%% diapo 21 %%%
%%%%%%%%%%%%%%
\begin{frame}
\frametitle{Cas pratique: visionnage de vidéos}


\end{frame}

%%%%%%%%%%%%%%
%%% diapo 22 %%%
%%%%%%%%%%%%%%
\begin{frame}
\frametitle{AT vs MP}

\textbf{Généralement, la relation de cause à effet est plus facile à établir pour un AT que pour une MP

AT = Causalité facile a établir} donc \textbf{prévention possible}
\begin{itemize}
\item fait matériel
\item constaté à un moment et à un endroit précis
\item entraînant un dommage instantané
\end{itemize}
\textbf{MP} = Imputation professionnelle sur des\textbf{ critères probabilistes de présomption} donc \textbf{prévention difficile}
\begin{itemize}
\item conséquence d'une exposition plus ou moins longue à un risque 
\begin{enumerate}
	\item connu ou non
\item pris en compte ou non
\item auquel la victime est toujours exposée (ou non) au moment de la première constatation
\end{enumerate}
\end{itemize}
\end{frame} 


%%%%%%%%%%%%%%
%%% diapo 23 %%%
%%%%%%%%%%%%%%
\begin{frame}
\frametitle{Les statistiques d’AT}
\framesubtitle{L’analyse d'un AT}

\end{frame} 

%%%%%%%%%%%%%%
%%% diapo 24 %%%
%%%%%%%%%%%%%%
\begin{frame}
\frametitle{Taux de fréquence/taux de gravité}

On utilise habituellement 2 critères pour analyser les accidents de travail par branche professionnelle et/ou par entreprise : 

\textbf{tableau}
\end{frame} 

%%%%%%%%%%%%%%
%%% diapo 25 %%%
%%%%%%%%%%%%%%
\begin{frame}
\frametitle{Taux de fréquence/taux de gravité}

Le taux de fréquence comme le taux de gravité n’ont \textbf{qu’un rapport indirect avec l’accident du travail.} 

Le taux de fréquence n’est surtout pas un taux de fréquence des accidents du travail et le taux de gravité n’est surtout pas un taux de gravité des accidents du travail.

\textbf{schéma}
\end{frame}


%%%%%%%%%%%%%%
%%% diapo 26 %%%
%%%%%%%%%%%%%%
\begin{frame}
\frametitle{Indices de fréquence}

\textbf{L’indice de fréquence }= nombre d’arrêts de travail de plus de 24 heures déclarés à la sécurité sociale, pour mille agents.

Son intérêt = + pratique pour effectuer des comparaisons entre organisations.

En effet, il est plus facile de connaître l’effectif  d’une structure que son nombre d’heures travaillées.


\textbf{schéma}
\end{frame}

%%%%%%%%%%%%%%
%%% diapo 27 %%%
%%%%%%%%%%%%%%
\begin{frame}
\frametitle{Accidents du Travail}
\framesubtitle{Taux de fréquence}

\textbf{schéma}
\end{frame}


%%%%%%%%%%%%%%
%%% diapo 28 %%%
%%%%%%%%%%%%%%
\begin{frame}
\frametitle{Indices de gravité}

\textbf{Indice de gravité} = somme des taux d’IPP pour un million d’heures travaillées.

Le taux d’IP (incapacités permanente) traduit la gravité des séquelles irréversibles subies par les victimes à la suite d’un accident.

L’indice de gravité \textbf{mesure ce qui coûte le plus cher} dans 1 AT.

\textbf{schéma}
\end{frame}

%%%%%%%%%%%%%%
%%% diapo 29 %%%
%%%%%%%%%%%%%%
\begin{frame}
\frametitle{Accidents du Travail}
\framesubtitle{Taux de gravité}

\textbf{schéma}
\end{frame}


%%%%%%%%%%%%%%
%%% diapo 30 %%%
%%%%%%%%%%%%%%
\begin{frame}
\frametitle{Un autre indicateur}

\textbf{Nombre de jours d’arrêt / Nombre d’arrêt de travail. }

Généralement lorsqu’une organisation «gère» les arrêts de travail : \textbf{le nombre moyen de jours d’arrêt de travail par arrêt de travail} a tendance à augmenter 
\begin{itemize}
\item \textit{réduction d’1 absentéisme de courte durée lié aux conditions de travail…?}
\end{itemize}
\end{frame}

%%%%%%%%%%%%%%
%%% diapo 31 %%%
%%%%%%%%%%%%%%
\begin{frame}
\frametitle{Accidents du Travail / Maladies professionnelles}
\framesubtitle{Quelques statistiques}

\begin{itemize}
\item  \textbf{22 \%} des accidents du travail concernent les moins de \textbf{25 ans} \textit{(1 jeune salarié sur 10 AT dans l’année)}
\item \textbf{760 000 évènements} dont 660 000 accidents du travail et 100 000 accidents de trajet
\item \textbf{54 millions de journées de travail perdues} à cause d’évènements survenus pendant le travail
\item \textbf{50 000} victimes de maladies professionnelles 
\item \textbf{Coût total = 7,5 milliards} d’\euro{}
\item \textbf{Coût moyen} d’\textbf{une maladie professionnelle = 23 421} \euro{}
\item \textbf{40 000 TMS} indemnisés
\item \textbf{Coût total} des MP liées aux \textbf{TMS = 787 millions} \euro{}
\item \textbf{8,4 millions de journées de travail perdues} 
\end{itemize}
\end{frame}

%%%%%%%%%%%%%%
%%% diapo 32 %%%
%%%%%%%%%%%%%%
\begin{frame}
\frametitle{Quelques chiffres du secteur BTP}

\textbf{D’après l’INRS, le BTP est le secteur d’activité qui présente le plus haut niveau de risque et qui déplore le plus grand nombre d’accidents graves et de décès}

\textbf{En 2012, BTP = secteur ayant le + haut niveau de risque}
\begin{itemize}
\item \textbf{Effectif salarié du BTP} : 8,6 \% de l’effectif total des salariés 
\item \textbf{Accidents mortels} : 25 \% des accidents mortels surviennent dans le BTP
\end{itemize}
\textbf{2 fois plus d’accidents de travail dans le BTP}

\textbf{Toutefois, au fil des années,  Indicateurs d’accidentologie dans le BTP = majoritairement à la baisse}
\begin{itemize}
\item Nombre total d’AT = - 6 \%
\item  Accidents avec arrêt = 640 891 soit - 4,3 \% 
\item  Accidents mortels = - 9 \%
\item  Accidents graves (ayant entraîné une incapacité permanente) = - 2,3 \%
\item  Nombre total de journées perdues pour incapacité temporaire = - 3,5 \%
\item  Taux de fréquence = - 3,3 \% (24,3 à 23,5 accidents par million d’heures travaillées)
\end{itemize}
\textbf{Nombre de décès = + 6 cas soit 558 décès }
\end{frame}

%%%%%%%%%%%%%%
%%% diapo 33 %%%
%%%%%%%%%%%%%%
\begin{frame}
\frametitle{Quelques chiffres}

\textbf{tableau}
\end{frame}

%%%%%%%%%%%%%%
%%% diapo 34 %%%
%%%%%%%%%%%%%%
\begin{frame}
\frametitle{Quelques chiffres}

\textbf{tableau}
\end{frame}

%%%%%%%%%%%%%%
%%% diapo 35 %%%
%%%%%%%%%%%%%%
\begin{frame}
\frametitle{Quelques chiffres – Secteur Privé}

\textbf{schéma}
\end{frame}

%%%%%%%%%%%%%%
%%% diapo 36 %%%
%%%%%%%%%%%%%%
\begin{frame}
\frametitle{Quelques chiffres – Secteur Privé}

\textbf{schéma}
\end{frame}

%%%%%%%%%%%%%%
%%% diapo 37 %%%
%%%%%%%%%%%%%%
\begin{frame}
\frametitle{Les conséquences administratives et financières de l'accident}
\framesubtitle{L’analyse d'un AT}

\end{frame}


%%%%%%%%%%%%%%
%%% diapo 38 %%%
%%%%%%%%%%%%%%
\begin{frame}
\frametitle{Les conséquences financières pour l’entreprise}

\textbf{Le coût des accidents du travail est divisé en deux parties : }
\begin{itemize}
\item \textbf{Le coût direct} correspond à la cotisation d’accident du travail et de maladie professionnelle,
\item \textbf{Le coût indirect} correspond à toutes les autres dépenses directement ou indirectement liées aux accidents du travail. 
\end{itemize}

\textbf{schéma}
\end{frame}

%%%%%%%%%%%%%%
%%% diapo 39 %%%
%%%%%%%%%%%%%%
\begin{frame}
\frametitle{Le coût des accidents du travail : le coût direct et indirect}

\textbf{schéma}
\end{frame}

%%%%%%%%%%%%%%
%%% diapo 40 %%%
%%%%%%%%%%%%%%
\begin{frame}
\frametitle{Détail du coût direct d’un accident du travail}

Les \textbf{dépenses versées à la CRAM (Caisse Primaire d’Assurance Maladie)} comprennent :
\begin{itemize}
\item Les indemnités journalières versées à l’accidenté 
\item Les frais médicaux 
\item Les frais de pharmacie
\item Eventuellement, les indemnités en capital ou des rentes allouées suite 	   à un accident.
\item Ces frais sont supportés par l’entreprise en fonction de son nombre de salariés (sauf dans pour les entreprises du BTP, les sièges sociaux et les bureaux, et les nouvelles entreprises qui ont un taux collectif).
\end{itemize}
 En effet, ces frais ci-dessus servent à calculer un taux de cotisation AT/MP . 
\end{frame}

%%%%%%%%%%%%%%
%%% diapo 41 %%%
%%%%%%%%%%%%%%
\begin{frame}
\frametitle{Détail du coût direct d’un accident du travail}

\begin{itemize}
\item \textbf{Les indemnités compensatrices} : Elles sont équivalentes à 40 % du salaire de l’accidenté.

\item \textbf{Le coût de remplacement} : C’est le remplacement de l’accidenté durant son arrêt de travail.
\end{itemize}

\textbf{Remarque} : A tous ces coûts il faut cependant enlever le salaire de l’accidenté, non versé pendant la durée de l’arrêt de travail.

\begin{itemize}
\item \textbf{La ou les visites de reprise} : La visite de reprise est obligatoire pour un arrêt de travail de plus de 8 jours. En cas de crainte d’inaptitude au travail ou/et d’aménagement de poste, une visite de pré reprise peut également s’avérer nécessaire.
\end{itemize}
\end{frame}

%%%%%%%%%%%%%%
%%% diapo 42 %%%
%%%%%%%%%%%%%%
\begin{frame}
\frametitle{Détails des coûts indirects des accidents du travail}

Ces coûts sont souvent plus abstraits et il est difficile de les quantifier.

\begin{itemize}
\item \textbf{Les coûts administratifs} : le détachement d’un employé administratif pour remplir les documents administratifs à envoyer à la CRAM, le temps passé à appeler l’inspection du travail... Dans ces coûts peuvent figurer également le temps passé à réaliser les enquêtes d’accident.

\item \textbf{Les pertes de production} : celles qui sont directement liées à l’arrêt du travail de l’accidenté, et on imagine également que lorsque l’accident du travail entraîne des pertes matérielles, la perte de production pendant le temps de la remise en service peut atteindre des sommes importantes.
\end{itemize}

Outre ces pertes, il peut subsister, plusieurs jours après l’accident, une démotivation des opérateurs, se traduisant par une baisse significative de la production, voire également une dégradation de la qualité.
\end{frame}

%%%%%%%%%%%%%%
%%% diapo 43 %%%
%%%%%%%%%%%%%%
\begin{frame}
\frametitle{Détails des coûts indirects des accidents du travail}

\begin{itemize}
\item \textbf{Les coûts matériels} : Ces coûts sont facilement chiffrables, il comprennent la remise en état, et/ou éventuellement la mise en conformité d’une machine ou d’un équipement.

\item \textbf{Les coûts commerciaux} : Ce sont les éventuelles pénalités de retard dues aux arrêts de production, on peut également imaginer une éventuelle perte de clientèle due aux retards de livraison, à la baisse de la qualité du produit suite à la démotivation des salariés.

\item \textbf{Les coûts répressifs} : Plus rares et en cas d’accident grave, l’entreprise peut se voir infliger des sanctions pénales.
\end{itemize}
\end{frame}

%%%%%%%%%%%%%%
%%% diapo 44 %%%
%%%%%%%%%%%%%%
\begin{frame}
\frametitle{Les conséquences juridiques}

\textbf{tableau}
\end{frame}

%%%%%%%%%%%%%%
%%% diapo 45 %%%
%%%%%%%%%%%%%%
\begin{frame}
\frametitle{Les conséquences juridiques }

\textbf{ATTENTION}

La déclaration d’AT à la CPAM est obligatoire, même en l’absence d’arrêt de travail. 
\begin{itemize}
\item En cas d’arrêt, l’employeur \textbf{transmet également à la CPAM} le formulaire « Attestation de salaire » (disponible à la CPAM) qui indique la période de travail, le nombre de journées et d’heures, le montant et la date de payes (article R.441-4 du code de la Sécurité sociale).
\item La CPAM peut en outre solliciter de l’employeur ou du salarié tous les renseignements qu’elle estime utiles.
\end{itemize}
\end{frame}

%%%%%%%%%%%%%%
%%% diapo 46 %%%
%%%%%%%%%%%%%%
\begin{frame}
\frametitle{Les conséquences juridiques }

\textbf{ATTENTION}
\begin{itemize}
\item L’employeur doit aussi \textbf{remettre au salarié victime une feuille d’accident} \textit{(article L.441-5 du code de la Sécurité sociale)}. 

\item Disponible à la CPAM, elle évite au salarié l’avance des frais dus aux soins médicaux. Cette feuille mentionne la caisse ayant en charge les prestations.

\item Elle ne doit pas indiquer au préalable le nom et l’adresse d’un praticien, d’un pharmacien, d’une clinique et elle est valable pour la durée du traitement résultant de l’accident \textit{(article R.441-8 du code de la Sécurité sociale). }

\item Le médecin et les auxiliaires médicaux, l’établissement hospitalier donnant les soins ainsi que le pharmacien remplissent cette feuille \textit{(article R.441-9 du code de la Sécurité sociale).}
\end{itemize}
\end{frame}

%%%%%%%%%%%%%%
%%% diapo 47 %%%
%%%%%%%%%%%%%%
\begin{frame}
\frametitle{Les conséquences juridiques}

\textbf{Dans le cadre d'1 AT/MP : La reconnaissance de la faute inexcusable engendre des conséquences financières pour l’entreprise}, notamment la majoration de la rente versée au salarié.

\textbf{La redéfinition de la faute inexcusable}

Depuis les arrêts du 28 février 2002, la cour de cassation a donné une nouvelle définition de la faute inexcusable de l’employeur.

La faute inexcusable de l’employeur est déterminée en cas de non-respect par celui-ci de l’obligation de sécurité de résultat découlant du contrat de travail et lorsque deux conditions sont réunies : 
\begin{itemize}
\item l’employeur avait ou aurait dû avoir conscience du danger risqué pour le salarié, 
\item l’employeur n’a pas pris les mesures nécessaires pour préserver le salarié. 
\end{itemize}
\end{frame}

%%%%%%%%%%%%%%
%%% diapo 48 %%%
%%%%%%%%%%%%%%
\begin{frame}
\frametitle{Les conséquences juridiques}

\textbf{Les indemnisations complémentaires :}
\begin{itemize}
\item Au-delà de la majoration de la rente, la victime peut prétendre à la réparation de préjudices*:  causé par des souffrances physiques et morales, esthétiques et d’agrément, ou encore lié à la diminution ou à la perte de ses probabilités de promotion professionnelle. 
\item En cas d’incapacité permanente de 100 \%, la victime bénéficie en + d’1 indemnité forfaitaire = au salaire minimum légal en vigueur à la date de consolidation (stabilisation de blessure résultant d’1 AT). 
\end{itemize}

\textbf{La réparation du préjudice moral pour les ayants-droits}
\begin{itemize}
\item Dans l’éventualité d’un accident ou d’une maladie professionnelle suivi du décès, les ayants droit de la victime ainsi que les ascendants et descendants ne pouvant pas prétendre à une rente peuvent demander à l’employeur réparation de leur préjudice moral \textit{(article L.452-3 du code de la sécurité sociale). }
\end{itemize}
\end{frame}

%%%%%%%%%%%%%%
%%% diapo 49 %%%
%%%%%%%%%%%%%%
\begin{frame}
\frametitle{Le rôle du CHSCT}

\textbf{schéma}
\end{frame}

%%%%%%%%%%%%%%
%%% diapo 50 %%%
%%%%%%%%%%%%%%
\begin{frame}
\frametitle{Le mécanisme de survenue de l’accident du travail}
\framesubtitle{L’analyse d'un AT}

\end{frame}

%%%%%%%%%%%%%%
%%% diapo 51 %%%
%%%%%%%%%%%%%%
\begin{frame}
\frametitle{Causes, événement et conséquences}

\textbf{tableau}
\end{frame}

%%%%%%%%%%%%%%
%%% diapo 52 %%%
%%%%%%%%%%%%%%
\begin{frame}
\frametitle{Causes, événement et conséquences}

\textbf{schéma}
\end{frame}

%%%%%%%%%%%%%%
%%% diapo 53 %%%
%%%%%%%%%%%%%%
\begin{frame}
\frametitle{Schéma Accident de travail}

\textbf{schéma}
\end{frame}

%%%%%%%%%%%%%%
%%% diapo 54 %%%
%%%%%%%%%%%%%%
\begin{frame}
\frametitle{Groupe de travail, enquête accident}
\framesubtitle{L’analyse d'un AT}

\end{frame}

%%%%%%%%%%%%%%
%%% diapo 55 %%%
%%%%%%%%%%%%%%
\begin{frame}
\frametitle{Groupe de travail}

\textbf{L'analyse d'un accident repose sur un travail de groupe, dont la structure doit être constituée de la façon suivante :}
\begin{itemize}
\item  L'encadrement de l'atelier ;
\item Des délégués du CHSCT ; 
\item Des membres du personnel de l'atelier ; 
\item La victime si cela est possible ; 
\item Le service sécurité.
\end{itemize} 

Avec l’ensemble des personnes concernées.

Certaines entreprises forment des personnes chargées de vérifier la conformité et la bonne application de la méthode.
\end{frame}

%%%%%%%%%%%%%%
%%% diapo 56 %%%
%%%%%%%%%%%%%%
\begin{frame}
\frametitle{Enquête accident}

\textbf{Le plus tôt possible après l’accident, car cela permet de :}
\begin{itemize}
\item Faciliter, pour les témoins et la personne accidentée, la description de l’accident;

\item Eviter d’émettre des hypothèses;

\item Garder le lieu de l’accident intact et ainsi permettre de relever des indices importants;

\item Identifier plus facilement les causes ayant contribué à l’accident;

\item Faciliter l’identification des mesures correctives;

\item Eviter qu’un accident semblable ne se répète.
\end{itemize} 
\end{frame}

%%%%%%%%%%%%%%
%%% diapo 57 %%%
%%%%%%%%%%%%%%
\begin{frame}
\frametitle{Enquête accident}

\textbf{Sur les lieux mêmes de l’accident : }
\begin{itemize}
\item Ceci permet aux personnes impliquées dans l’accident de mieux expliquer ce qui s’est produit et aux enquêteurs de mieux comprendre ;

\item Les gens sont moins enclins à s’éloigner de la réalité lorsqu’ils sont sur les lieux mêmes de l’accident; dans le cas contraire, certains ont tendance à modifier le rôle joué par les facteurs environnants ;

\item En allant sur les lieux, les enquêteurs peuvent être en mesure d’apporter des correctifs temporaires ou permanents pour éviter toute répétition de l’accident ;

\item Ceci permet également aux enquêteurs de décrire avec exactitude le lieu où l’accident est survenu. Bien situer physiquement le lieu où l’événement s’est déroulé peut être utile pour le dénouement de l’enquête. Très souvent, le lieu de l’accident fournit autant d’informations sur les causes de l’événement que l’entrevue de témoins.
\end{itemize} 
\end{frame}

%%%%%%%%%%%%%%
%%% diapo 58 %%%
%%%%%%%%%%%%%%
\begin{frame}
\frametitle{Méthodes d’analyses}
\framesubtitle{QQOQCP / arbre des causes / ITAMAMI / faire le point (RPS)}
\end{frame}

%%%%%%%%%%%%%%
%%% diapo 59 %%%
%%%%%%%%%%%%%%
\begin{frame}
\frametitle{Rechercher les déterminants}
\framesubtitle{Méthode QQOQCP}

\end{frame}

%%%%%%%%%%%%%%
%%% diapo 60 %%%
%%%%%%%%%%%%%%
\begin{frame}
\frametitle{Arbre des causes}

\end{frame}


%%%%%%%%%%%%%%
%%% diapo 61 %%%
%%%%%%%%%%%%%%
\begin{frame}
\frametitle{L’arbre des causes}

La méthode de l'arbre des causes (arbre de défaillance) est une technique très largement éprouvée dans le domaine des risques professionnels. Cette méthodologie développée dans les années 1970 par l'INRS permet d’\textbf{organiser à partir du recueil des faits le plan complet de l’accident}.
\begin{itemize}
\item Il s’agit d’une approche objective qui permet de relier schématiquement les facteurs accidentogènes (causes) et l’accident du travail (effet). 
\end{itemize}

Contrairement à une démarche ergonomique préventive \textit{a priori} (l'incident ou l'accident n'a pas encore eu lieu, on évalue le rapport entre le travail prescrit et le travail réel afin d'en tirer les conséquences pour améliorer la situation de travail), l’arbre des causes est une \textbf{démarche préventive a posteriori (l'incident ou l'accident a eu lieu, on recherche les causes de cet événement afin que cela ne puisse plus se reproduire)}. 
\begin{itemize}
\item l’objectif est de déterminer la totalité des causes, les mettre en parallèle les unes par rapport aux autres et enfin de trouver des solutions à chacune de ces causes (l’idée sous-jacente étant que la suppression d'une des causes entrainant \textit{de facto} la suppression de l'accident). 
\end{itemize}
\textit{Monteau, M. (1974). Méthode pratique de recherche des facteurs d'accidents. Principes et application expérimentale. Rapport INRS, 140(RE), 1-68.}
\end{frame}

%%%%%%%%%%%%%%
%%% diapo 62 %%%
%%%%%%%%%%%%%%
\begin{frame}
\frametitle{Principe de construction}

L’arbre des causes est une représentation graphique des faits recueillis
Il se construit de la droite vers la gauche, à partir du fait ultime
Le but est de rechercher, étape par étape, l’organisation des faits qui ont concouru à l’accident.
\begin{itemize}
\item Les 2 étapes pour construire un arbre des causes
\begin{enumerate}
	\item Recueillir les faits
\item Organiser les faits
\end{enumerate}
\end{itemize}
\end{frame}


%%%%%%%%%%%%%%
%%% diapo 63 %%%
%%%%%%%%%%%%%%
\begin{frame}
\frametitle{Recueillir les faits}

\textbf{Qu’est-ce qu’un fait ?}
\begin{itemize}
	\item Un \textbf{fait} est un élément ayant concouru à l’accident.
\item Il s’agit d’une information \textbf{objective, vérifiable et précise}.
\item Les \textbf{opinions et interprétations} ne doivent donc pas être retenus parmi les faits.
\end{itemize}
\end{frame}

%%%%%%%%%%%%%%
%%% diapo 64 %%%
%%%%%%%%%%%%%%
\begin{frame}
\frametitle{Un fait, c’est quoi ?}

\textbf{Un fait est une information :}
\begin{itemize}
	\item Vérifiable
\item Non contestable
\item Concrète
\item Concise
\end{itemize}
Pour éviter tout \textbf{jugement et toute interprétation}, il faut toujours s’efforcer de s’en tenir à :
\begin{itemize}
	\item Qui à fait quoi ?
\item Comment ?
\item Avec qui ?
\item Où ?
\item Quand ?
\end{itemize}
\end{frame}

%%%%%%%%%%%%%%
%%% diapo 65 %%%
%%%%%%%%%%%%%%
\begin{frame}
\frametitle{Distinction entre un fait, une opinion, une interprétation, un jugement}

\textbf{tableau}
\end{frame}

%%%%%%%%%%%%%%
%%% diapo 66 %%%
%%%%%%%%%%%%%%
\begin{frame}
\frametitle{Distinction entre un fait, une opinion, une interprétation, un jugement}

\textbf{schéma}
\end{frame}

%%%%%%%%%%%%%%
%%% diapo 67 %%%
%%%%%%%%%%%%%%
\begin{frame}
\frametitle{Recueillir les faits}

\textbf{Auprès de qui recueillir les faits ?}
\begin{itemize}
	\item La victime
\item Les collègues de travail
\item Le service de maintenance qui gère le matériel
\item Le service sécurité
\item Le CHSCT
\end{itemize}
\end{frame}

%%%%%%%%%%%%%%
%%% diapo 68 %%%
%%%%%%%%%%%%%%
\begin{frame}
\frametitle{Recueillir les faits}

\textbf{Comment recueillir les faits ?}

En questionnant les personnes sur :
\begin{itemize}
	\item L’individu (la victime)
\item La tâche qu’il effectuait au moment de l’accident
\item Le matériel utilisé
\item Le milieu de travail
\end{itemize}
\end{frame}

%%%%%%%%%%%%%%
%%% diapo 69 %%%
%%%%%%%%%%%%%%
\begin{frame}
\frametitle{Exemple}

\textbf{Pour notre exemple, après le recueil des faits, voici l’histoire de l’accident :}
\begin{itemize}
	\item Paul, mécanicien qualifié, a été appelé à 11 h pour réparer une centrifugeuse de l’atelier C.
\item Cette réparation était urgente car une \textbf{commande importante} devait être prête le soir même.
\item L’atelier entretien se trouve dans le bâtiment A. 
\item Le chemin le plus court et habituel longe le bâtiment B dont la toiture était en réfection. 
\item L’accès de ce passage était interdit par un panneau de signalisation mais non barré matériellement. 
\item Le jour de l’accident, le vent soufflait depuis le matin en violentes rafales. 
\item Paul, \textbf{tête nue, passait} avec sa caisse à outils au pied du bâtiment B, quand il a reçu sur la tête une \textbf{tuile tombant} de la toiture. 
\item Le \textbf{choc} a occasionné une profonde \textbf{blessure à la tête} qui a nécessité le transport de la victime à l’hôpital.
\end{itemize}
\end{frame}

%%%%%%%%%%%%%%
%%% diapo 70 %%%
%%%%%%%%%%%%%%
\begin{frame}
\frametitle{Recueillir les faits}

Récit de l’accident obtenu grâce aux témoignages :
\begin{itemize}
	\item Paul, mécanicien, \textbf{passait, tête nue}, au pied du bâtiment B, quand il a reçu sur la \textbf{tête une tuile tombant de la toiture}. Le \textbf{choc} a occasionné une \textbf{profonde blessure à la tête}

\item \textbf{Parmi les faits préalablement listés, repérer le fait ultime correspondant à la blessure de l’opérateur.}
\end{itemize}
\end{frame}

%%%%%%%%%%%%%%
%%% diapo 71 %%%
%%%%%%%%%%%%%%
\begin{frame}
\frametitle{Recueillir les faits}

\textbf{Récit de l’accident obtenu grâce aux témoignages :}
\begin{itemize}
	\item Paul, mécanicien, \textbf{passait, tête nue}, au pied du bâtiment B, quand il a reçu sur la tête une \textbf{tuile tombant de la toiture}. Le \textbf{choc} a occasionné une \textbf{profonde blessure à la tête}.
	\end{itemize}
\end{frame}

%%%%%%%%%%%%%%
%%% diapo 72 %%%
%%%%%%%%%%%%%%
\begin{frame}
\frametitle{Organiser les faits}

\textbf{Il faut se poser systématiquement 3 questions à chaque étape :}
\begin{itemize}
	\item Qu’a-t-il fallu pour que ce fait se réalise ?
\item Est-ce nécessaire?
\item Est-ce suffisant ?
\end{itemize}
\end{frame}

%%%%%%%%%%%%%%
%%% diapo 73 %%%
%%%%%%%%%%%%%%
\begin{frame}
\frametitle{Organiser les faits}

Il s’agit de reconstruire les enchaînements et les combinaisons des faits qui ont joué un rôle dans la survenue de l’accident.
Pour construire l’arbre des causes, on part du ou des faits ultimes (ou de l’incident) que l’on veut analyser et l’on remonte systématiquement, pas à pas, en se posant pour chaque fait que l’on trouve 2 questions :
\begin{itemize}
\item qu’a-t-il fallu pour ?
\item est-ce suffisant ?
\end{itemize}
schéma
\end{frame}

%%%%%%%%%%%%%%
%%% diapo 74 %%%
%%%%%%%%%%%%%%
\begin{frame}
\frametitle{Organiser les faits}

\textbf{Le fait}

schéma
\end{frame}


%%%%%%%%%%%%%%
%%% diapo 75 %%%
%%%%%%%%%%%%%%
\begin{frame}
\frametitle{Organiser les faits}

Cependant, on arrête souvent l’enquête trop rapidement en se limitant aux faits évidents, proches de l’accident ou de l’incident.

On a intérêt à \textbf{rechercher les causes profondes} de l’accident en\textbf{ remontant le plus en amont} possible.

schéma
\end{frame}


%%%%%%%%%%%%%%
%%% diapo 76 %%%
%%%%%%%%%%%%%%
\begin{frame}
\frametitle{Organiser les faits}

Le fait \textit{« choc de la tuile sur la tête »} a 3 causes.

Dans ce cas, les faits s’organisent sous forme d’une \textbf{conjonction} :

schéma
\end{frame}

%%%%%%%%%%%%%%
%%% diapo 77 %%%
%%%%%%%%%%%%%%
\begin{frame}
\frametitle{L’arbre des causes}

\textbf{schéma}
\end{frame}


%%%%%%%%%%%%%%
%%% diapo 78 %%%
%%%%%%%%%%%%%%
\begin{frame}
\frametitle{Organiser les faits}

Il existe un dernier type de lien lorsqu’on a un seul antécédent pour plusieurs faits.

La \textbf{disjonction} :

schéma
\end{frame}

%%%%%%%%%%%%%%
%%% diapo 79 %%%
%%%%%%%%%%%%%%
\begin{frame}
\frametitle{Organiser les faits}


\textbf{schéma}
\end{frame}

%%%%%%%%%%%%%%
%%% diapo 80 %%%
%%%%%%%%%%%%%%
\begin{frame}
\frametitle{Représentation graphique de l'enchaînement des causes}

\textbf{schéma}
\end{frame}

%%%%%%%%%%%%%%
%%% diapo 81 %%%
%%%%%%%%%%%%%%
\begin{frame}
\frametitle{Organiser les faits}

\textbf{schéma}
\end{frame}

%%%%%%%%%%%%%%
%%% diapo 82 %%%
%%%%%%%%%%%%%%
\begin{frame}
\frametitle{Intérêt de l’arbre des causes}

\textbf{Identifier les causes de l’accident.}

Cet arbre des causes sera utilisé, dans un deuxième temps afin de proposer des \textbf{moyens de prévention} adaptés pour qu’un tel accident ne se reproduise pas.
\end{frame}

%%%%%%%%%%%%%%
%%% diapo 83 %%%
%%%%%%%%%%%%%%
\begin{frame}
\frametitle{Exploitation de l’arbre des causes et choix des mesures de prévention}
\framesubtitle{L’analyse d'un AT}

\end{frame}

%%%%%%%%%%%%%%
%%% diapo 84 %%%
%%%%%%%%%%%%%%
\begin{frame}
\frametitle{Exploitation de l’arbre des causes}

\textbf{L’arbre des causes fait apparaître un enchaînement de faits nécessaires à la survenue d’un accident.}

L’arbre des causes permet de proposer des mesures de prévention en recherchant à tous les niveaux les possibilités d’action capables d’empêcher la production de l’accident.

Pour ce faire :
\begin{itemize}
\item On examine systématiquement tous les faits de l’arbre
\item On recherche systématiquement pour chacun d’entre eux s’il existe un ou plusieurs moyens de le supprimer, d’en empêcher l’apparition et d’en éviter les conséquences néfastes.
\end{itemize}
\end{frame}

%%%%%%%%%%%%%%
%%% diapo 85 %%%
%%%%%%%%%%%%%%
\begin{frame}
\frametitle{Pour éviter le renouvellement d’un autre accident, nous recherchons
des mesures préventives pour chaque fait de l’arbre des causes.}

\textbf{tableau}
\end{frame}

%%%%%%%%%%%%%%
%%% diapo 86 %%%
%%%%%%%%%%%%%%
\begin{frame}
\frametitle{Les critères à prendre en compte}

\textbf{Généralement, pour faire les choix entres diverses propositions d’actions de prévention, on prend en compte certains critères essentiels :}
\begin{itemize}
\item Conformité à la réglementation
\item Coût pour l’entreprise
\item Stabilité de la mesure
\item Portée de la mesure
\begin{enumerate}
	\item \textit{La mesure envisagée a-t-elle uniquement une application locale ou est-elle susceptible de résoudre un problème de sécurité ailleurs ?}
\end{enumerate}
\item Non-déplacement du risque
\begin{enumerate}
	\item \textit{La mesure envisagée a-t-elle uniquement une application locale ou est-elle susceptible de résoudre un problème de sécurité ailleurs ?}
\end{enumerate}
\end{itemize}
\end{frame}

%%%%%%%%%%%%%%
%%% diapo 87 %%%
%%%%%%%%%%%%%%
\begin{frame}
\frametitle{Les critères à prendre en compte}

\textbf{Généralement, pour faire les choix entres diverses propositions d’actions de prévention, on prend en compte certains critères essentiels :}
\begin{itemize}
\item Non-déplacement 
\begin{enumerate}
\item \textit{La mesure de prévention envisagée – localement bénéfique – ne risque-t-elle pas d’entraîner des répercutions néfastes ailleurs ?}
\end{enumerate}
\item Coût pour l’opérateur
\begin{enumerate}
\item \textit{La mesure envisagée a-t-elle uniquement une application locale ou est-elle susceptible de résoudre un problème de sécurité ailleurs ?}
\end{enumerate}
\item Non-déplacement La mesure envisagée entraînera-t-elle une modification dans le travail susceptible d’augmenter la charge de travail des opérateurs concernés ?
\item Délai d’application…
\end{itemize}
\end{frame}

%%%%%%%%%%%%%%
%%% diapo 88 %%%
%%%%%%%%%%%%%%
\begin{frame}
\frametitle{Présentation d’une grille de critères de choix}

\textbf{schéma}
\end{frame}

%%%%%%%%%%%%%%
%%% diapo 89 %%%
%%%%%%%%%%%%%%
\begin{frame}
\frametitle{Evaluation des solutions retenues en fonction de leur niveau de prévention}

\textbf{Les solutions retenues sont évaluées en fonction : }
\begin{itemize}
\item \textbf{Elimination} de la situation dangereuse à la source
\item \textbf{Diminution} du risque par protection
\begin{enumerate}
\item protection à la source
\item protection collective
\item protection individuelle
\end{enumerate}
\item Maintien de la situation dangereuse
\begin{enumerate}
\item information, formation, consignes...
\item \textbf{1 - 2 - 3 par ordre d’importance}
\end{enumerate}
\end{itemize}
\end{frame}

%%%%%%%%%%%%%%
%%% diapo 90 %%%
%%%%%%%%%%%%%%
\begin{frame}
\frametitle{Evaluation des solutions retenues en fonction de leur niveau de prévention}

\textit{En réunion, nous argumenterons les solutions proposées :}
\begin{itemize}
\item En privilégiant les critères suivants :
\begin{enumerate}
\item \textbf{Supprimer effectivement le risque}
\item  \textbf{Ne pas engendrer de surcharge pour l’opérateur}
\item \textit{Ne pas créer d’autres risques}
\item  \textbf{Mesurer durablement dans le temps}
\end{enumerate}

\item En optant pour une diversité de solutions ITAMAMI
\end{itemize}
\end{frame}

%%%%%%%%%%%%%%
%%% diapo 91 %%%
%%%%%%%%%%%%%%
\begin{frame}
\frametitle{Evaluation des solutions retenues en fonction de leur niveau de prévention}

\textbf{article}
\end{frame}

%%%%%%%%%%%%%%
%%% diapo 92 %%%
%%%%%%%%%%%%%%
\begin{frame}
\frametitle{Cas pratique :}

Analyser un accident du travail survenu avec la méthode de l’arbre des causes et proposer des mesures de prévention.
\end{frame}

%%%%%%%%%%%%%%
%%% diapo 93 %%%
%%%%%%%%%%%%%%
\begin{frame}
\frametitle{Suivre l’application des solutions}
\framesubtitle{L’analyse d'un AT}

\end{frame}

%%%%%%%%%%%%%%
%%% diapo 94 %%%
%%%%%%%%%%%%%%
\begin{frame}
\frametitle{La mise en œuvre et le suivi}

La mise en œuvre et le suivi des solutions sont facilités par une organisation ; nous proposons un outil de gestion.

\textbf{La mise en oeuvre :}
\begin{itemize}
\item \textbf{qui est chargé de la réalisation ?}
\item \textbf{dans quels délais ?}
\item \textbf{quels moyens seront nécessaires ?}
\end{itemize}
\textbf{Le suivi des mesures :}
\begin{itemize}
\item \textbf{Qui dans l’équipe suit et surveille l’avancement des réalisations ?}
\item  \textbf{Qui se propose d’informer tous les participants des conclusions de cette étude d’accident ?}
\end{itemize}
\end{frame}

%%%%%%%%%%%%%%
%%% diapo 95 %%%
%%%%%%%%%%%%%%
\begin{frame}
\frametitle{ITAMAMI}


\end{frame}

%%%%%%%%%%%%%%
%%% diapo 96 %%%
%%%%%%%%%%%%%%
\begin{frame}
\frametitle{La situation de travail « I TA MA MI »}

\textbf{schéma}
\end{frame}


%%%%%%%%%%%%%%
%%% diapo 97 %%%
%%%%%%%%%%%%%%
\begin{frame}
\frametitle{I TA MA MI}

\textbf{schéma}
\end{frame}

%%%%%%%%%%%%%%
%%% diapo 98 %%%
%%%%%%%%%%%%%%
\begin{frame}
\frametitle{Le recueil des faits}

La qualité de l’analyse repose sur la qualité des données qu’elle examine. C’est pourquoi le recueil des informations relatives à l’accident aura la plus grande importance.

\textbf{Recueillir des faits concrets et objectifs et non pas des interprétations et des jugements de valeurs.}

\end{frame}

%%%%%%%%%%%%%%
%%% diapo 99 %%%
%%%%%%%%%%%%%%
\begin{frame}
\frametitle{Auprès de qui recueillir des faits ?}

De toute personne susceptible d’apporter des informations sur la situation de travail : Individu, Tâche, Matériel et Milieu :

\begin{itemize}
\item la victime : ce qu’elle faisait ;
\item les collègues : ils connaissent le travail ;
\item le service maintenance : il connaît l’état du matériel ;
\item le service sécurité ;
\item le CHSCT ;
\item le médecin ;
\item …/…
\end{itemize}
\end{frame}

%%%%%%%%%%%%%%
%%% diapo 100 %%%
%%%%%%%%%%%%%%
\begin{frame}
\frametitle{Quelles questions poser ?}

\textbf{L'Individu :}
\begin{itemize}
\item Quelle expérience avait-il du travail effectué ?
\item Pour quelles raisons agissait-il comme cela ?
\item Avait-il des difficultés particulières ?
\end{itemize}

\textbf{La tâche au moment de l'accident :}
\begin{itemize}
\item Que faisait-il effectivement au moment de l’accident ?
\item Comment s’y prenait-il ?
\item Pour quelles raisons devait-il faire ce travail de cette façon ?
\item Y a-t-il eu quelque chose d’inhabituel : incident de dysfonctionnement ?
\item Y a-t-il eu des modifications par rapport au mode opératoire habituel ?
\end{itemize}
\end{frame}

%%%%%%%%%%%%%%
%%% diapo 101 %%%
%%%%%%%%%%%%%%
\begin{frame}
\frametitle{Quelles questions poser ?}

\textbf{Le matériel :}
\begin{itemize}
\item Pour quelles raisons utilisait-il ce matériel ?
\item Quelle machine, quels outils utilisait-il ?
\item Quels sont les risques connus sur cet outillage, machine ?
\item Dans quel état était ce matériel ? (entretien, maintenance, vétusté)
\item Y a-t-il eu une panne, une défaillance ? Laquelle ?
\end{itemize}

\textbf{Le milieu :}
\begin{itemize}
\item Quelle était l’organisation du travail ?
\item  Y a-t-il eu des facteurs d’ambiance physique, chimique... qui ont joué ?
\item  Quelles communications dans le travail ? quels moyens, qualité....?
\item  Qualité des relations dans l’équipe ?
\end{itemize}
\end{frame}

%%%%%%%%%%%%%%
%%% diapo 102 %%%
%%%%%%%%%%%%%%
\begin{frame}
\frametitle{Quand recueillir les faits ?}

\textbf{Immédiatement après l’accident « à chaud », 
mais également postérieurement} pour compléter les informations.
\end{frame}

%%%%%%%%%%%%%%
%%% diapo 103 %%%
%%%%%%%%%%%%%%
\begin{frame}
\frametitle{Exo pratique : I TA MA MI}

\textbf{Consignes :}
\begin{itemize}
\item Nous vous proposons un texte relatant un accident du travail.
\item Vous en dégagerez les différents faits.
\item Vous les classerez suivant les quatre critères de la méthode I-TA-MA-MI.
\end{itemize}
\end{frame}

%%%%%%%%%%%%%%
%%% diapo 104 %%%
%%%%%%%%%%%%%%
\begin{frame}
\frametitle{Exo pratique : I TA MA MI}

\textbf{schéma}

\end{frame}

%%%%%%%%%%%%%%
%%% diapo 105 %%%
%%%%%%%%%%%%%%
\begin{frame}
\frametitle{Exo pratique : I TA MA MI}

\textbf{schéma}
\end{frame}

%%%%%%%%%%%%%%
%%% diapo 106 %%%
%%%%%%%%%%%%%%
\begin{frame}
\frametitle{Exo pratique : I TA MA MI}

\textbf{schéma}
\end{frame}

%%%%%%%%%%%%%%
%%% diapo 107 %%%
%%%%%%%%%%%%%%
\begin{frame}
\frametitle{Exo pratique : I TA MA MI}

\textbf{schéma}
\end{frame}

%%%%%%%%%%%%%%
%%% diapo 108 %%%
%%%%%%%%%%%%%%
\begin{frame}
\frametitle{Exo pratique : I TA MA MI}

\textbf{schéma}
\end{frame}

%%%%%%%%%%%%%%
%%% diapo 109 %%%
%%%%%%%%%%%%%%
\begin{frame}
\frametitle{Exo pratique : I TA MA MI}

\textbf{schéma}
\end{frame}

%%%%%%%%%%%%%%
%%% diapo 110 %%%
%%%%%%%%%%%%%%
\begin{frame}
\frametitle{Exo pratique : I TA MA MI}

\textbf{schéma}
\end{frame}

%%%%%%%%%%%%%%
%%% diapo 111 %%%
%%%%%%%%%%%%%%
\begin{frame}
\frametitle{Exo pratique : I TA MA MI}

\textbf{schéma}
\end{frame}

%%%%%%%%%%%%%%
%%% diapo 112 %%%
%%%%%%%%%%%%%%
\begin{frame}
\frametitle{Exo pratique : I TA MA MI}

\textbf{schéma}
\end{frame}

%%%%%%%%%%%%%%
%%% diapo 113 %%%
%%%%%%%%%%%%%%
\begin{frame}
\frametitle{Exo pratique : I TA MA MI}

\textbf{schéma}
\end{frame}

%%%%%%%%%%%%%%
%%% diapo 114 %%%
%%%%%%%%%%%%%%
\begin{frame}
\frametitle{Exo pratique : I TA MA MI}

\textbf{schéma}
\end{frame}

%%%%%%%%%%%%%%
%%% diapo 115 %%%
%%%%%%%%%%%%%%
\begin{frame}
\frametitle{Exo pratique : I TA MA MI}

\textbf{schéma}
\end{frame}

%%%%%%%%%%%%%%
%%% diapo 116 %%%
%%%%%%%%%%%%%%
\begin{frame}
\frametitle{Exo pratique : I TA MA MI}

\textbf{schéma}
\end{frame}

%%%%%%%%%%%%%%
%%% diapo 117 %%%
%%%%%%%%%%%%%%
\begin{frame}
\frametitle{Exo pratique : I TA MA MI}

\textbf{schéma}
\end{frame}

%%%%%%%%%%%%%%
%%% diapo 118 %%%
%%%%%%%%%%%%%%
\begin{frame}
\frametitle{Exo pratique : I TA MA MI}

\textbf{schéma}
\end{frame}

%%%%%%%%%%%%%%
%%% diapo 119 %%%
%%%%%%%%%%%%%%
\begin{frame}
\frametitle{Exo pratique : I TA MA MI}

\textbf{schéma}
\end{frame}

%%%%%%%%%%%%%%
%%% diapo 120 %%%
%%%%%%%%%%%%%%
\begin{frame}
\frametitle{Exo pratique : I TA MA MI}

\textbf{schéma}
\end{frame}

%%%%%%%%%%%%%%
%%% diapo 121 %%%
%%%%%%%%%%%%%%
\begin{frame}
\frametitle{Faire le point}
\framesubtitle{Un outil pour l'analyse des RPS}

\end{frame}

%%%%%%%%%%%%%%
%%% diapo 122 %%%
%%%%%%%%%%%%%%
\begin{frame}
\frametitle{Analyse des RPS}

L’outil « \textbf{Faire le point} »
\begin{itemize}
\item Développé par l’INRS, permet aux entreprises de s’interroger sur la présence ou non de RPS en répondant à une quarantaine de questions. Il fournit des clés de compréhension et des pistes d’actions pour les prévenir.
\item En fonction des résultats obtenus avec cet outil « Faire le point », et notamment pour les entreprises n’ayant pas obtenu de consensus interne, une démarche complémentaire est proposée avec l’outil.
\end{itemize}

L’outil « \textbf{Analyse des situations-problème} »
\begin{itemize}
\item Issu d’une méthodologie ANACT, propose aux entreprises de partir de situations réelles pour identifier les causes et conséquences des problèmes et ainsi trouver dans l’organisation de l’entreprise les clés d’amélioration.
\end{itemize}
\end{frame}
\end{document}