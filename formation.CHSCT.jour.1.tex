\documentclass{beamer}
\usepackage[utf8]{inputenc}
\usepackage[T1]{fontenc}
\usepackage[french]{babel}
\usepackage{eurosym}

% These slides also contain speaker notes. You can print just the slides,
% just the notes, or both, depending on the setting below. Comment out the want
% you want.
%\setbeameroption{hide notes} % Only slides
%\setbeameroption{show only notes} % Only notes
\setbeameroption{show notes on second screen=right} % Both
% To give a presentation with the Skim reader (http://skim-app.sourceforge.net) on OSX so
% that you see the notes on your laptop and the slides on the projector, do the following:
% 
% 1. Generate just the presentation (hide notes) and save to slides.pdf
% 2. Generate onlt the notes (show only nodes) and save to notes.pdf
% 3. With Skim open both slides.pdf and notes.pdf
% 4. Click on slides.pdf to bring it to front.
% 5. In Skim, under "View -> Presentation Option -> Synhcronized Noted Document"
%    select notes.pdf.
% 6. Now as you move around in slides.pdf the notes.pdf file will follow you.
% 7. Arrange windows so that notes.pdf is in full screen mode on your laptop
%    and slides.pdf is in presentation mode on the projector.
% Give a slight yellow tint to the notes page
% \setbeamertemplate{note page}{\pagecolor{yellow!5}\insertnote}\usepackage{palatino}

\begin{document}

%%%%%%%%%%%%%%
%%% diapo 1 %%%
%%%%%%%%%%%%%%
\begin{frame}
\frametitle{Jour 1}
\framesubtitle{Formation des membres du CHSCT}
\end{frame}

%%%%%%%%%%%%%%
%%% diapo 2 %%%
%%%%%%%%%%%%%%
\begin{frame}
\frametitle{Présentation}
\begin{itemize}
\item Présentation de la formation

\item Présentation des participants

\begin{enumerate}
        \item Leur travail, leur poste/leur fonction,…
        \item Leurs attentes, leurs besoins,…
\end{enumerate}
\end{itemize}
\end{frame}



%%%%%%%%%%%%%%
%%% diapo 3 %%%
%%%%%%%%%%%%%%

\begin{frame}
\frametitle{Les objectifs de le formation}

\begin{itemize}
\item \textbf{Maîtriser} les règles de fonctionnement du CHSCT

\item \textbf{Identifier} les moyens dont il dispose

\item \textbf{Clarifier} les missions de prévention qui résultent des dispositions légales en vigueur

\item Savoir \textbf{détecter} les risques professionnels

\item Contribuer à l’amélioration des conditions de travail

\item \textbf{Réaliser} efficacement les missions d’inspection et d’enquête

\item \textbf{Proposer} des actions concrètes pour contribuer à la prévention
\end{itemize}
\end{frame}  

%%%%%%%%%%%%%%
%%% diapo 4 %%%
%%%%%%%%%%%%%%

\begin{frame}
\frametitle{Introduction}
\end{frame} 


%%%%%%%%%%%%%%
%%% diapo 5 %%%
%%%%%%%%%%%%%%

\begin{frame}
\frametitle{Historique des CHSCT}
\begin{itemize}
\item \textbf{1941} : Création des premiers Comités de Sécurité 

\item \textbf{1941} : Création des premiers Comités de Sécurité 

\item \textbf{1947} : Ils deviennent des Comités d Hygiène et de Sécurité (CHS)

\item \textbf{1973} : Création des Commissions pour l'Amélioration des Conditions  de Travail (CACT dans les entreprises de + 300 salariés)

\item \textbf{1982} : Fusion des CHS et des CACT en Comité d'Hygiène, de Sécurité et des Conditions de Travail (CHSCT) 

\item \textbf{1993} : Décret n. 93X449 du 23 mars 1993 précisant les dispositions concernant les CHSCT, complété
\end{itemize}
\end{frame} 

%%%%%%%%%%%%%%
%%% diapo 6 %%%
%%%%%%%%%%%%%%
\begin{frame}
\frametitle{Acteurs de la prévention : national}

\begin{itemize}
\item La CRAM/\textbf{CARSAT} = Caisse d'assurance retraite et de la santé au travail 

\item Les organismes sous tutelle du ministère chargé du travail : 
\begin{enumerate}
        \item \textbf{ANACT} = agence nationale pour l’amélioration des conditions de travail,
        \item \textbf{OPPBTP} = organisme professionnel de prévention du bâtiment et des travaux publics
\end{enumerate}

\item L’\textbf{InVS} = institut de veille sanitaire

\item L’\textbf{INRS} 

\item Autres : Observatoires nationaux,…
\end{itemize}
\end{frame} 

%%%%%%%%%%%%%%
%%% diapo 7 %%%
%%%%%%%%%%%%%%
\begin{frame}
\frametitle{Acteurs de la prévention : régional}

\begin{itemize}
\item Les \textbf{CARSAT}

\item Les \textbf{DIRECCTE} 

\item Les ARACT = association régionale  de l’amélioration des conditions de travail (\textit{ex.: Act Méditerranée, ARAVIS, etc.)}

\item \textbf{L’échelon régional de l’OPPBTP}

\item Les \textbf{Instituts et Sociétés de Médecine du travail}

\item Les \textbf{Observatoires régionaux}

\item \textbf{Création des CRPRP} = comités régionaux des prévention des risques professionnels (prévus par le plan Santé au travail 2005-2009 
\end{itemize}
\end{frame} 

%%%%%%%%%%%%%%
%%% diapo 8 %%%
%%%%%%%%%%%%%%
\begin{frame}
\frametitle{Acteurs de la prévention : territorial}
\begin{itemize}
\item L’\textbf{inspection du travail}

\item Les \textbf{services de santé au travail}
\end{itemize}
\end{frame} 

%%%%%%%%%%%%%%
%%% diapo 9 %%%
%%%%%%%%%%%%%%
\begin{frame}
\frametitle{Acteurs de la prévention : entreprise}
\begin{itemize}
\item L’\textbf{employeur}

\item Les instances représentatives du personnel (IRP)


\begin{enumerate}
        \item \textbf{Les délégués du personnel}
        \item \textbf{Le CHSCT}
        \item \textbf{Les CE}
        \item \textbf{Le médecin du travail}
\end{enumerate}
\item Les IPRP

\item Autres : infirmières d’entreprise, secouristes, pompiers d’entreprise,...
\end{itemize}
\end{frame} 

%%%%%%%%%%%%%%
%%% diapo 10 %%%
%%%%%%%%%%%%%%
\begin{frame}
\frametitle{Champs d’application des CHSCT}
\framesubtitle{(circulaire n. 93-15 du 25 mars 1993)}

\textbf{En fonction de l’activité}

\textbf{Sont concernés} : Tous les établissements et entreprises de droit privé (\textbf{y compris le BTP}), les établissements industriels, commerciaux et agricoles publics ainsi que les établissements sanitaires et sociaux publics (Art. L.4111-1  du code du travail).
\textbf{Ne sont pas soumises à ces dispositions :}
\begin{itemize}
        \item les mines et carrières et leurs dépendances, 
        \item les entreprises de transport par fer, par route, par eau et par air dont les institutions particulières ont été fixées par voie statutaire.
\end{itemize}
\textbf{Obligatoire dans tous les établissements} sup ou égal \textbf{50 salariés.}
\begin{itemize}
        \item L'effectif doit être atteint pendant 12 mois, consécutifs ou non, au cours des 3 dernières années précédentes. Il est calculé mensuellement en tenant compte de l'ensemble des travailleurs (Art. L4611-1 du Code du travail).
        \item Si impossibilité de créer un CHSCT (carence de candidatures, les DP exercent les attributions du CHSCT et disposent des mêmes moyens.
\end{itemize}   
\end{frame}

%%%%%%%%%%%%%%
%%% diapo 11 %%%
%%%%%%%%%%%%%%
\begin{frame}
\frametitle{Champs d’application des CHSCT}
\framesubtitle{(circulaire n. 93-15 du 25 mars 1993)}

\textbf{En fonction de l’effectif}

\textbf{Etablissements} sup ou égal \textbf{50 salariés :}
\begin{itemize}
        \item \textbf{DP} = investis des missions dévolues aux membres du CHSCT, mais cette fois dans le cadre de leurs moyens propres (sauf dispositions conventionnelles plus favorables). 
        \item \textbf{Inspecteur du travail} = peut imposer la création d'un CHSCT notamment en raison de la nature des travaux effectués dans l'établissement, de l'agencement ou de l'équipement des locaux.
        \item \textbf{Possibilité de se grouper} sur le plan professionnel ou inter-professionnel pour créer un CHSCT.
\end{itemize}
\end{frame} 


%%%%%%%%%%%%%%
%%% diapo 12 %%%
%%%%%%%%%%%%%%
\begin{frame}
\frametitle{Les IRP prévues par le Droit Français}
\begin{itemize}
\item Les  \textbf{D.P.}

\item Le  \textbf{C.E.}

\item Le  \textbf{C.H.S.C.T.}

\item Les  \textbf{D.S.}
\end{itemize}
\end{frame}

%%%%%%%%%%%%%%
%%% diapo 13 %%%
%%%%%%%%%%%%%%
\begin{frame}
\frametitle{Les textes du Droit social}
\begin{itemize}
\item Le \textbf{CODE DU TRAVAIL}

\item Les \textbf{CONVENTIONS COLLECTIVES}

\item Les \textbf{ACCORDS D’ENTREPRISE}

\item Le \textbf{CONTRAT DE TRAVAIL}

\item Les \textbf{JURISPRUDENCES}

\item Les \textbf{USAGES}
\end{itemize}
\end{frame}

%%%%%%%%%%%%%%
%%% diapo 14 %%%
%%%%%%%%%%%%%%
\begin{frame}
\frametitle{Les Obligations du CHSCT}

\textbf{La confidentialité}

\textit{«  Les membres du comité sont tenus à une \textbf{obligation de discrétion} à l'égard des informations présentant un caractère confidentiel et données comme telles par le chef d'établissement ou son représentant. » Article L4614-9.}

Il faut, au secrétaire du CHSCT, être vigilant sur la question de la confidentialité des informations. 

Pour être plus clair : il faut \textbf{s’interdire de mentionner sur ses procès-verbaux, les noms des victimes d’accident, de maladie professionnelle ou autre, de retranscrire des propos de nature injurieuse ou diffamatoire}.
\end{frame} 

%%%%%%%%%%%%%%
%%% diapo 15 %%%
%%%%%%%%%%%%%%
\begin{frame}
\frametitle{Les Obligations du CHSCT}

\textbf{Le secret professionnel}

« Les membres du CHSCT sont tenus au secret professionnel pour toutes les questions relatives aux procédés de fabrication. »

Le non respect du secret professionnel est, lui, un délit qui est réprimé par l’article 226-13 du Code Pénal :

\begin{itemize}
        \item \textit{« La révélation d'une information à caractère secret par une personne qui en est dépositaire soit par état ou par profession, soit en raison d'une fonction ou d'une mission temporaire, est punie d'un an d'emprisonnement et de 15 000 \euro{} d'amende. »}
\end{itemize}
\end{frame}

%%%%%%%%%%%%%%
%%% diapo 16 %%%
%%%%%%%%%%%%%%
\begin{frame}
\frametitle{Définition du droit d'alerte}

\textbf{Principe}

\textit{Droit « de se retirer d’une situation de travail dont (…  on a un motif raisonnable de penser qu’elle présente un danger grave et imminent pour sa vie ou sa santé ».}

\begin{itemize}
        \item \textbf{Gravité du danger} : \textit{« un danger susceptible de produire un accident ou une maladie entraînant la mort ou paraissant devoir entraîner une incapacité permanente ou temporaire prolongée. »} Circulaire du 25 mars 1993.
       
        \item \textbf{Imminence du danger} : \textit{« tout danger susceptible de se réaliser brutalement dans un délai rapproché »} Circulaire du 25 mars 1993.
\end{itemize}
\end{frame}

%%%%%%%%%%%%%%
%%% diapo 17 %%%
%%%%%%%%%%%%%%
\begin{frame}
\frametitle{Description du processus}

\textbf{schéma}
\end{frame}

%%%%%%%%%%%%%%
%%% diapo 18 %%%
%%%%%%%%%%%%%%
\begin{frame}
\frametitle{La procédure mise en place}

\textbf{schéma}
\end{frame}

%%%%%%%%%%%%%%
%%% diapo 19 %%%
%%%%%%%%%%%%%%
\begin{frame}
\frametitle{Création et mise en place d'un CHSCT}

\end{frame}

%%%%%%%%%%%%%%
%%% diapo 20 %%%
%%%%%%%%%%%%%%
\begin{frame}
\frametitle{Finalité et attribution du CHSCT}
\framesubtitle{Création et mise en place d'un CHSCT}

\end{frame}


%%%%%%%%%%%%%%
%%% diapo 21 %%%
%%%%%%%%%%%%%%
\begin{frame}
\frametitle{Le but du CHSCT}

\textbf{schéma}
\end{frame}

%%%%%%%%%%%%%%
%%% diapo 22 %%%
%%%%%%%%%%%%%%
\begin{frame}
\frametitle{Les missions du CHSCT}

\textbf{Domaines de compétence}

\begin{itemize}
        \item Hygiène, sécurité, conditions de travail. 
 
        \item Organisation matérielle du travail (charge, pénibilité).  
 
        \item Aménagement des lieux de postes de travail (ergonomie).  
 
        \item Durée, temps, rythme de travail (organisation). 
 
        \item Environnement physique du travail (exposition).
\end{itemize}

\textbf{Objectifs}

\begin{itemize}
        \item Contribuer à la protection de la santé physique et mentale et de la sécurité des personnels de l'organisation et des travailleurs mis à disposition.

        \item Veiller à l'observation des prescriptions légales. 

        \item Promouvoir la prévention des risques professionnels.

         \item Contribuer à l'amélioration des conditions de travail.
\end{itemize}
\end{frame}

%%%%%%%%%%%%%%
%%% diapo 23 %%%
%%%%%%%%%%%%%%
\begin{frame}
\frametitle{Les attributions du CHSCT (1/3)}

\textbf{Consultation}

\begin{itemize}
        \item Le CHSCT peut être saisi par l'employeur, le CE ou les DP.

        \item Le CHSCT doit être consulté et se prononcé sur toute question et tout document relevant de sa compétence.
\end{itemize}

\textbf{Consultation obligatoire}

\begin{itemize}
        \item Avant toute décision d'aménagement important modifiant les conditions de santé et de sécurité ou les conditions de travail.

        \item Sur les projets d'introduction et lors de l'introduction de nouvelles technologies.

        \item Sur le plan d'adaptation en cas de mutations technologiques importantes et rapides.

        \item Sur la mise, la remise ou le maintien au travail des accidentés du travail, des invalides et travailleurs handicapés.

        \item Sur les modalités  pratiques d'élaboration du document unique et sur le plan d'action qui en découle.
\end{itemize}
\end{frame}

%%%%%%%%%%%%%%
%%% diapo 24 %%%
%%%%%%%%%%%%%%
\begin{frame}
\frametitle{Les attributions du CHSCT (2/3)}

\textbf{Avis : donne un avis (mesures supplémentaires, priorité, etc...)  sur :}

\begin{itemize}
        \item Le rapport annuel sur la situation générale de la santé, de la sécurité et des conditions du travail et des actions menées.
        \item Le programme annuel de prévention des risques professionnels et d'amélioration des conditions de travail.
\end{itemize}

\textbf{Actions de terrain}

\begin{itemize}
\item Inspections régulières (au moins tous les 3 mois). 

        \item Enquêtes en cas d'accident ou de maladie professionnelle.

        \item Etude et analyse des risques professionnels de l'entreprise.
\end{itemize}
\end{frame}

%%%%%%%%%%%%%%
%%% diapo 25 %%%
%%%%%%%%%%%%%%
\begin{frame}
\frametitle{Les attributions du CHSCT (3/3)}

\textbf{Situation de danger grave et imminent :}

\begin{itemize}
        \item Alerter l'employeur.

        \item Consigner dans un registre spécial.

\item Mettre en œuvre une enquête immédiate et des mesures de prévention.

\item En cas de divergence, réunion extraordinaire du CHSCT.

        \item En cas de désaccord persistant, saisie l'inspecteur du travail.
\end{itemize}
\end{frame}

%%%%%%%%%%%%%%
%%% diapo 26 %%%
%%%%%%%%%%%%%%
\begin{frame}
\frametitle{Création et composition du CHSCT}
\framesubtitle{Création et mise en place d'un CHSCT}
\end{frame}

%%%%%%%%%%%%%%
%%% diapo 27 %%%
%%%%%%%%%%%%%%
\begin{frame}
\frametitle{La création du CHSCT}

\begin{itemize}
\item \textbf{Obligatoire} dans tous les établissements sup ou égal \textbf{50 salariés}.

\item En général, la \textbf{désignation} se fait \textbf{par consensus}.

\item Les membres sont tenus à l'\textbf{obligation de secret professionnel et de discrétion}.

\item Le \textbf{non respect} du secret professionnel est un \textbf{délit puni} par l’article 225-13 du nouveau code pénal.
\end{itemize}
\end{frame}

%%%%%%%%%%%%%%
%%% diapo 28 %%%
%%%%%%%%%%%%%%
\begin{frame}
\frametitle{La Composition du CHSCT}
\framesubtitle{Art. L.4613-1 du Code du travail}

\textbf{Siègent au CHSCT :}

\begin{itemize}
        \item  Le chef d'établissement ou son représentant assumant la présidence.

\item Lors de la création du CHSCT le chef d’établissement ne participe pas à cette réunion, son rôle est de réunir le collège désignatif du CHSCT.

\item Une délégation du personnel.

\item A titre consultatif, le médecin du travail et le chef du service de sécurité et des conditions de travail (Art R. 4614-2)
  
\item L'inspecteur du travail et l'agent du service de prévention de la CRAM peuvent assister aux réunions (Art. L.4614-11, R 4614-3 et 4). 

\item Le CHSCT peut faire appel à titre consultatif et occasionnel au concours de toute personne de l'établissement qui lui paraîtrait qualifiée sur un thème particulier (infirmière, responsable technique, architecte...

\item Etablissements sup ou égal 300 salariés, les organisations syndicales représentatives peuvent désigner un représentant syndical au CHSCT siégeant alors avec voix consultative.
\end{itemize}
\end{frame}

%%%%%%%%%%%%%%
%%% diapo 29 %%%
%%%%%%%%%%%%%%
\begin{frame}
\frametitle{La Composition du CHSCT}
\framesubtitle{Art. L.4613-1 du Code du travail}

\textbf{La délégation du personnel}
\begin{itemize}
\item Membres - salariés de l'entreprise - \textbf{désignés pour 2 ans} (Art. R.4613-5) \textbf{renouvelables}

\item \textbf{Élu par un collège} constitué par les \textbf{membres élus du CE et les DP}. En cas de démission d'un élu, le remplaçant ne peut être désigné que par ce même collège. 

\item \textbf{Critères d’inclusion} \textit{(définis par l'Art L 2324-15 du Code du travail)} pour les membres du CE  :
\begin{enumerate}
        \item Ne pas appartenir à la famille du chef d'entreprise, 

\item Avoir 18 ans accomplis, 

\item Travailler dans l'entreprise sans interruption depuis 1 an, 

\item Ne pas avoir été déchu de ses fonctions syndicales, ni avoir été condamné pour indignité nationale. 
\end{enumerate}
\item Le mandat peut se cumuler avec celui de membre du CE, de DP, de DS.
\end{itemize}
\end{frame}

%%%%%%%%%%%%%%
%%% diapo 30 %%%
%%%%%%%%%%%%%%
\begin{frame}
\frametitle{La Composition du CHSCT}

\textbf{schéma}
\end{frame}

%%%%%%%%%%%%%%
%%% diapo 31 %%%
%%%%%%%%%%%%%%
\begin{frame}
\frametitle{Le détail de la composition du CHSCT}

\textbf{schéma}
\end{frame}

%%%%%%%%%%%%%%
%%% diapo 32 %%%
%%%%%%%%%%%%%%
\begin{frame}
\frametitle{La présidence du CHSCT}


\begin{itemize}
        \item Le C.H.S.C.T. est présidé par le chef d’établissement ou son représentant.

\item Il doit avoir les pouvoirs nécessaires pour présider.
\end{itemize}
\end{frame}

%%%%%%%%%%%%%%
%%% diapo 33 %%%
%%%%%%%%%%%%%%
\begin{frame}
\frametitle{Les pouvoirs du Président}

\textbf{La délégation du personnel}

Le Président du CHSCT est dans une position particulière car il est et reste avant toute chose le \textbf{chef d’établissement ou son représentant}. La façon dont le président va s’acquitter de son rôle va avoir un gros impact sur la qualité du fonctionnement du CHSCT.
\begin{itemize}
\item \textbf{En aucun cas le Président du CHSCT ne peut agir au nom du CHSCT}, à moins d’avoir été dûment mandaté à cet effet. Il ne peut pas prendre une décision sur le fonctionnement du CHSCT contre l‘avis de la majorité des membres présents. 

\item Quand il peut voter, le Président ne dispose pas non plus d’\textbf{une voix prépondérante}*. Par ailleurs, dans certains cas, il ne doit absolument pas voter : avis sur le règlement intérieur, sur le rapport et le programme annuel, sur les projets modifiant les conditions de travail, etc… 

\item Comme chef d’établissement, \textbf{il peut agir contre le CHSCT}, par exemple pour attaquer, devant le TGI*, une décision dont il conteste la légalité \textit{(ex. recours à une expertise votée en CHSCT)}.

\item Le Président \textbf{assume seul la responsabilité de convoquer le CHSCT}, 15j avant la réunion. Il doit s’assurer que les convocations soient bien parvenues à l’ensemble des destinataires (membres élus, médecin / inspection du travail, CARSAT.

\item \textbf{L’ordre du jour des réunions est rédigé conjointement par le Président et le secrétaire CHSCT}, chacun apportant les points qu’il souhaite à l’ordre du jour. Le Président ne peut s’opposer à ce qu’un point particulier soit porté à l’ordre du jour.
\end{itemize}
\end{frame}

%%%%%%%%%%%%%%
%%% diapo 34 %%%
%%%%%%%%%%%%%%
\begin{frame}
\frametitle{Les devoirs d’information du Président}
\begin{itemize}
\item Sauf à commettre un délit d’entrave au fonctionnement régulier du CHSCT, le \textbf{Président du CHSCT à le devoir d’informer le CHSCT}. 
\begin{enumerate}
\item L’article L. 4614-9 précise que \textit{« le CHSCT reçoit du chef d’établissement les informations qui lui sont nécessaires pour l’exercice de ses missions.»}
\end{enumerate}
\item Le président doit en outre informer le CHSCT \textbf{sur les contrôles et vérifications techniques obligatoires des installations, des machines ou des équipements de travail.}

\begin{enumerate}
\item Présentation des documents établis par les organismes externes au CHSCT.
\end{enumerate}
\item Le chef d’établissement doit tenir\textbf{ un registre des mises en demeure formulées par l’inspecteur du travail} et présenter ce registre à tout membre du CHSCT qui en ferait la demande. Il doit également informer le CHSCT \textbf{des éventuelles observations de l’inspecteur du travail, du médecin du travail et de l’agent du service prévention de la CARSAT} dès la réunion suivant ces observations.

\item Le chef d’établissement doit tenir à la disposition des membres du CHSCT le \textbf{document unique} dans lequel il retranscrit les résultats de l'évaluation des risques pour la sécurité et la santé des travailleurs.
\end{itemize}
\end{frame}

%%%%%%%%%%%%%%
%%% diapo 35 %%%
%%%%%%%%%%%%%%
\begin{frame}
\frametitle{Les devoirs de consultation du Président}

 
\begin{itemize}
\item Le Président a une \textbf{obligation légale de consultation} du CHSCT, sur les documents se rattachant à sa mission.

\item\textbf{ La consultation peut porter sur} : le règlement intérieur de l’établissement, les consignes de sécurité, les notes de services ou tout autre document qui instituerait des règles générales et permanentes dans le domaine de l’hygiène et de la sécurité.

\item Le Président du CHSCT doit consulter le CHSCT avant \textbf{toute décision d’aménagement important modifiant les conditions d'hygiène et de sécurité ou les conditions de travail} et, notamment, avant toute transformation importante des postes de travail découlant de la modification de l'outillage, d'un changement de produit ou de l'organisation du travail, avant toute modification des cadences et des normes de productivité liées ou non à la rémunération du travail.

\item Le Président doit consulter le CHSCT sur les mesures prises en vue de faciliter \textbf{la mise, la remise ou le maintien au travail des accidentés du travail}, des invalides de guerre, des invalides civils et des travailleurs handicapés, notamment sur l'aménagement des postes de travail.

\item Le Président doit consulter chaque année le CHSCT sur \textbf{le rapport faisant le bilan de la situation en matière d’hygiène, de sécurité et des conditions de travail} et concernant les actions qui ont été menées au cours de l’année écoulée, ainsi que sur le\textbf{ programme annuel de prévention} des risques professionnels et \textbf{d’amélioration} des conditions de travail.
\end{itemize}
\end{frame}

%%%%%%%%%%%%%%
%%% diapo 36 %%%
%%%%%%%%%%%%%%
\begin{frame}
\frametitle{La désignation du secrétaire du CHSCT}
\begin{itemize}
\item Pour un CHSCT qui se forme, \textbf{l’élection de son secrétaire est un acte important}. Le \textbf{1er vote} important du CHSCT est généralement consacré à cette élection interne, lors de la \textbf{1ère réunion} qui suit la désignation des représentants du personnel. 

\item L’élection du secrétaire du CHSCT aura donc naturellement lieu à \textbf{la majorité des membres présents}. 

\item \textbf{En cas d’égalité}, la règle du code électoral prévaudra et c’est le \textbf{candidat le plus âgé} qui sera déclaré secrétaire du CHSCT. 
\begin{enumerate}
\item S’abstenir, dans les votes du CHSCT, ou même ne pas voter, revient à voter non…
\end{enumerate}
\item Dans de nombreuses entreprises, l’usage est que le Président laisse les représentants du personnel désigner eux-mêmes leur secrétaire.
\end{itemize}
\end{frame}

%%%%%%%%%%%%%%
%%% diapo 37 %%%
%%%%%%%%%%%%%%
\begin{frame}
\frametitle{Le rôle du secrétaire de CHSCT (1/2)}

\begin{itemize}
\item \textbf{Porte Parole} des représentants du personnel au CHSCT auprès du Comité d'entreprise.

\item \textbf{Fixe avec le Président} du CHSCT l'ordre du jour des réunions.

\item \textbf{Etablit} le procès-verbal des réunions.

\item \textbf{Interlocuteur principal} du chef d'établissement, il parle au nom de la majorité des élus.
\end{itemize}
\end{frame}


%%%%%%%%%%%%%%
%%% diapo 38 %%%
%%%%%%%%%%%%%%
\begin{frame}
\frametitle{Le rôle du secrétaire de CHSCT (2/2)}

\begin{itemize}
\item Le secrétaire du CHSCT n’a pas voix prépondérante, il ne peut, ni ne doit rien imposer à ses pairs. 

\item Il doit jouer un \textbf{rôle d’animateur et d’organisateur de l’activité du CHSCT}. 

\item Il a mandat pour fixer, avec le chef d’établissement, \textbf{l’ordre du jour des réunions}. Il a aussi mandat pour \textbf{établir le PV des réunions}. Le secrétaire du CHSCT \textbf{administre les affaires courantes du comité}, c’est en fait l’interlocuteur principal du chef d’établissement, il parle au nom de la majorité des élus. 

\item Souvent le secrétaire se verra donner des \textbf{mandats particuliers} pour agir au nom du comité. 

\item \textbf{Le code du travail ne fixe pas de durée au mandat} du secrétaire. La question de son maintien ou de son remplacement, portée à un ordre du jour suffit pour le maintenir ou le remplacer par un vote à la majorité des membres présents. Pour la cohérence de l’action du CHSCT, et pour assister le secrétaire dans ses tâches, il peut être utile de nommer un \textbf{secrétaire adjoint}.
\end{itemize}
\end{frame}

%%%%%%%%%%%%%%
%%% diapo 39 %%%
%%%%%%%%%%%%%%
\begin{frame}
\frametitle{Les missions du secrétaire de CHSCT}

\textbf{Le point juridique sur la question}
\begin{itemize}
\item L’article L. 4614-8 prévoit dans son dernier alinéa que \textbf{le secrétaire du CHSCT est désigné parmi les représentants du personnel} et que \textbf{l’ordre du jour} de la réunion est \textbf{établi conjointement} par le \textbf{secrétaire} et le \textbf{président du CHSCT}.

\item \textbf{L’article R. 4614-3} précise les \textbf{modalités de communication} de l’ordre du jour et des pièces jointes.

\item \textbf{L’article R. 4614-4} précise les conditions de conservation des procès-verbaux de réunion, du rapport et du programme mentionné à l’article L. 4612-16.
\end{itemize}
\end{frame}

%%%%%%%%%%%%%%
%%% diapo 40 %%%
%%%%%%%%%%%%%%
\begin{frame}
\frametitle{Le rôle du secrétaire de CHSCT}

\textbf{Le point juridique sur la question : Article L. 4614-3}
\textit{« L'ordre du jour de la réunion du CHSCT et, le cas échéant, les documents s'y rapportant sont transmis par le président aux membres du comité et à l'inspecteur du travail} \textbf{15 jours au moins avant} \textit{la date fixée pour} \textbf{la réunion}, \textit{sauf cas exceptionnel justifié par l'urgence}.

\textit{Toutefois, lorsque le comité est réuni dans le cadre d'un projet de restructuration et de compression des effectifs mentionné à l'article L. 2323-15, l'ordre du jour et, le cas échéant, les documents s'y rapportant sont transmis} \textbf{3 jours au moins avant} \textit{la date fixée pour la} \textbf{réunion}.

\textit{L'ordre du jour est transmis dans les mêmes conditions aux agents des services de prévention des organismes de sécurité sociale qui peuvent assister aux réunions du comité. »}
\end{frame}

%%%%%%%%%%%%%%
%%% diapo 41 %%%
%%%%%%%%%%%%%%
\begin{frame}
\frametitle{Le rôle du secrétaire de CHSCT}

\textbf{Le point juridique sur la question : Article L. 4614-4}

\textit{« Les} \textbf{réunions du CHSCT} \textit{ont lieu} \textbf{dans l'établissement, dans un local approprié} \textit{et, sauf exception justifiée par l'urgence,} \textbf{pendant les heures de travail.}

\textit{Les} \textbf{PV} \textit{des réunions ainsi que} \textbf{le rapport et le programme annuels} \textit{mentionnés à l'article L. 4612-16 sont} \textbf{conservés dans l'établissement}. 

\textit{Ils sont} \textbf{tenus à la disposition de l'inspecteur du travail, du médecin inspecteur du travail et des agents des services de prévention des organismes de sécurité sociale. »}
\end{frame}

%%%%%%%%%%%%%%
%%% diapo 42 %%%
%%%%%%%%%%%%%%
\begin{frame}
\frametitle{Les Moyens du secrétaire du CHSCT}

\begin{itemize}
\item Le secrétaire du CHSCT reçoit du chef d’établissement les moyens de réaliser les missions qui lui sont confiées. 

\item L’article L. 4614-9 du Code du Travail \textit{(n. 2008-67 du 21 janvier 2008)} \textbf{prévoit que le comité reçoit les moyens nécessaires à la préparation et à l'organisation des réunions}. 

\item Ces moyens doivent notamment comprendre, au minimum, \textbf{les moyens de dactylographie nécessaires, de reproduction, de transmission et de diffusion des PV} \textit{(ex : panneaux d'affichage, ou tout autre moyen adéquat de diffusion)} \textbf{et une documentation juridique et technique adaptée aux risques particuliers de l'établissement}.

\item La loi n’impose pas à l’employeur de mettre un local à la disposition exclusive du CHSCT. Pourtant, \textbf{le secrétaire du CHSCT doit disposer d’un endroit où classer ses documents et archives}. Les représentants du personnel doivent disposer d’une salle équipée lorsqu’ils ont besoin de se réunir et de travailler collectivement. La sagesse des juges a amené la jurisprudence à définir le niveau d’équipement du comité en rapport avec celui des autres services de l’établissement.
\end{itemize}
\end{frame}

%%%%%%%%%%%%%%
%%% diapo 43 %%%
%%%%%%%%%%%%%%
\begin{frame}
\frametitle{Les moyens du CHSCT}

\textbf{Faire appel à un expert interne}
\begin{itemize}
\item Ou à toute personne  qualifiée de l’entreprise.
\end{itemize}

\textbf{Faire appel à des experts agréés}
 
\begin{itemize}
\item La mission est définie par le CHSCT.

\item L’expert obtient le droit d’entrée dans l’établissement et l’accès aux informations nécessaires à la mission.

\item L’expert est tenu à une obligation de discrétion et de respect du secret professionnel.

\item Les frais engagés sont à la charge de l’employeur.
\end{itemize}
\textbf{Formation}
\begin{itemize}
\item Formation nécessaire à l’exercice de la mission.
\begin{enumerate}
\item Coûts à la charge de l’employeur: frais de déplacement, d’hébergement.

\item Formation dispensée par un organisme agréé par le ministère du travail.

\item Renouvelable tous les 4 ans.
\end{enumerate}
\end{itemize}
\textbf{Personne civile}
\begin{itemize}
\item Le CHSCT peut agir en justice
\end{itemize}
\end{frame}

%%%%%%%%%%%%%%
%%% diapo 44 %%%
%%%%%%%%%%%%%%
\begin{frame}
\frametitle{Informations indispensables à l'exercice de ses missions}
\begin{itemize}
\item Chaque membre du CHSCT peut demander que lui soit communiqué le \textbf{registre} destiné à l'inscription des \textbf{mises en demeure} de l'inspection du travail.

\item Doivent être présentés au CHSCT les \textbf{registres} tenus en application des prescriptions réglementaires imposant \textbf{des vérifications périodiques de certains appareils ou machines (Article R. 4614-5)}.

\item En cas d’\textbf{intervention d'une entreprise extérieure} (Art. R. 4515-1, à R4515-10), le décret du 20 février 1992 prévoit pour l'entreprise extérieure l'obligation d'informer son CHSCT mais aussi celui de l'entreprise utilisatrice :
\begin{enumerate}
\item Informations relatives à la durée des interventions prévues, aunombre de salariés affectés à ces interventions. 

\item Références des sous-traitants éventuels et identification des travaux sous-traités. 

\item Date de l'inspection préalable et dates de réunions de coordination.
\end{enumerate}
\end{itemize}
\end{frame}

%%%%%%%%%%%%%%
%%% diapo 45 %%%
%%%%%%%%%%%%%%
\begin{frame}
\frametitle{Les informations constituent une matière première indispensable…}

Informations prévues par un texte 
les textes sur les machines (R. 4323-2) :
\begin{itemize}
\item les textes sur les équipements individuels de protection (R. 4323-97),

\item les registres de vérification (R.46146-5, L.4711-1 à 3),

\item le règlement intérieur et les consignes (L1321-4, L 1321-5),

\item les courriers de l’inspection du travail, du service prévention de la CRAM, du médecin du travail,

\item les informations sur un projet avant consultation du CHSCT…
\end{itemize}

Informations obtenues en référence à l’article L4614-9 du Code du Travail :

\textit{« CHSCT} \textbf{reçoit de l'employeur} \textit{les informations qui lui sont} \textbf{nécessaires} \textit{pour l'exercice de ses missions, ainsi que les} \textbf{moyens nécessaires} \textit{à la préparation et à l'organisation des réunions et aux déplacements imposés par les enquêtes ou inspections.} 

Les \textbf{membres du CHSCT} \textit{sont tenus à une} \textbf{obligation de discrétion} \textit{à l'égard des informations présentant un caractère confidentiel et données comme telles par l'employeur.} 

\textbf{Ils sont tenus} \textbf{au secret professionnel} \textit{pour toutes les questions relatives aux procédés de fabrication ».}
\end{frame}

%%%%%%%%%%%%%%
%%% diapo 46 %%%
%%%%%%%%%%%%%%
\begin{frame}
\frametitle{Informations obligatoires}

\textbf{schéma}
\end{frame}


%%%%%%%%%%%%%%
%%% diapo 47 %%%
%%%%%%%%%%%%%%
\begin{frame}
\frametitle{Consultations obligatoires}

\textbf{schéma}
\end{frame}


%%%%%%%%%%%%%%
%%% diapo 48 %%%
%%%%%%%%%%%%%%
\begin{frame}
\frametitle{Recours à l’expertise}
\framesubtitle{(Art. L4614-12, Art. R4614-6 à R4614-17)}
\textbf{schéma}
\end{frame}

%%%%%%%%%%%%%%
%%% diapo 49 %%%
%%%%%%%%%%%%%%
\begin{frame}
\frametitle{Recours à l’expertise}
\framesubtitle{(Art. L4614-12, Art. R4614-6 à R4614-17)}

\textbf{Une procédure réglementée}

\begin{itemize}
\item Le Code du travail prévoit un \textbf{formalisme bien précis}, que les élus CHSCT doivent respecter avec soin pour éviter toute contestation ou retard possible de l’expertise. 

\item Souhaitable de \textbf{contacter « en amont » le cabinet expert} pressenti qui pourra \textbf{vous assister} dans tout le processus et \textbf{vous aider} à rédiger les différents documents. 
\end{itemize}
\textbf{Objectifs de la mission}

\begin{itemize}
\item D’\textbf{analyser} les \textbf{situations de travail actuelles} ainsi que le(s) \textbf{projet(s) de transformation} afin d’établir un \textbf{diagnostic} des transformations prévues ou en cours et un\textbf{ pronostic de leurs effets sur les conditions de travail et la santé} des salariés.

\item \textbf{D’aider le CHSCT à avancer des propositions} de prévention des risques professionnels et d’amélioration des conditions de travail.
\end{itemize}
\end{frame}

%%%%%%%%%%%%%%
%%% diapo 50 %%%
%%%%%%%%%%%%%%
\begin{frame}
\frametitle{La formation}

\textbf{schéma}
\end{frame}


%%%%%%%%%%%%%%
%%% diapo 51 %%%
%%%%%%%%%%%%%%
\begin{frame}
\frametitle{Le crédit d’heures}

\textbf{schéma}
\end{frame}

%%%%%%%%%%%%%%
%%% diapo 52 %%%
%%%%%%%%%%%%%%
\begin{frame}
\frametitle{Libre circulation}

\textbf{Droit de libre circulation dans tout l’établissement}
\end{frame}

%%%%%%%%%%%%%%
%%% diapo 53 %%%
%%%%%%%%%%%%%%
\begin{frame}
\frametitle{La protection contre le licenciement}

\begin{itemize}
\item Les salariés représentants du personnel ne peuvent faire l’objet \textbf{d’un licenciement}, individuel ou collectif, \textbf{sans l’autorisation de l’inspecteur du travail, pendant toute la durée de leur mandat et au-delà.

\item L’inspecteur du travail vérifie au cours d’une enquête contradictoire} que la rupture du contrat n’est pas une mesure discriminatoire, liée aux fonctions de représentation du salarié. Sa décision, positive ou négative, peut faire l’objet d’un recours.

\item \textbf{Tout salarié candidat lors d’une élection professionnelle, titulaire ou ancien titulaire d’un mandat de représentant du personnel bénéficie d’une protection} contre la rupture de son contrat de travail. 
\begin{enumerate}
\item Bénéficient également de la protection contre le licenciement, \textbf{le DS, le DP, le membre du CE, le représentant du personnel au CHSCT}, institués par convention ou accord collectif de travail.
\end{enumerate}
\end{itemize}
\end{frame}

%%%%%%%%%%%%%%
%%% diapo 54 %%%
%%%%%%%%%%%%%%
\begin{frame}
\frametitle{La réunion : 4 types}

\textbf{Les réunions périodiques :}

\begin{itemize}
\item Doivent avoir lieu au moins tous les trimestres.
\item L’initiative de réunir le CHSCT appartient à l’employeur.
\end{itemize}

\textbf{Les réunions extraordinaires :}
\begin{itemize}
\item A la demande d’au moins 2 membres du CHSCT.

\item L’employeur n’a pas à apprécier le bien fondé des motifs.
\end{itemize}

\textbf{Les réunions suite à un accident :}
\begin{itemize}
\item Accident ayant entraîné ou qui aurait pu entraîner des conséquences graves.

\item Toute cause d’accident pouvant se répéter justifie la convocation du comité quelles qu’aient été les conséquences.
\end{itemize}

\textbf{Les réunions suite à la mise en danger :}
\begin{itemize}
\item Dans le cadre de la procédure d’alerte, le CHSCT est réuni d’urgence dans les 24 heures.
\end{itemize}
\end{frame}

%%%%%%%%%%%%%%
%%% diapo 55 %%%
%%%%%%%%%%%%%%
\begin{frame}
\frametitle{L’ordre du jour : élément essentiel et réglementé}

\begin{itemize}
\item \textbf{L’ordre du jour} est le déroulé de la réunion. 

\item Il est \textbf{rédigé conjointement} par le secrétaire et le président, ils ne peuvent pas le rédiger seul. 

\item Il est \textbf{joint à la convocation}. 

\item En réunion, \textbf{seuls les points figurant à l’ordre du jour doivent être traités}. De ce fait, le recours à un point dit « questions diverses » est à proscrire.
\end{itemize}
\end{frame}

%%%%%%%%%%%%%%
%%% diapo 56 %%%
%%%%%%%%%%%%%%
\begin{frame}
\frametitle{L’ordre du jour : élément essentiel et réglementé}

\textbf{Il peut se composer ainsi…}

\begin{itemize}
\item Approbation du PV de la réunion précédente.

\item Avis sur les projets et aménagements.

\item Examen du rapport annuel et du programme annuel de prévention.

\item Observation de l’inspection du travail, de l’inspecteur des installations classées, et/ou de l’agent de service prévention de la CARSAT…

\item Observations des membres du CHSCT sur les documents et les registres.

\item Informations que l’employeur doit donner au CHSCT.

\item Compte rendu des inspections des locaux de travail.

\item Point sur les missions individuelles confiées à des membres du CHSCT.

\item Point sur la formation à la sécurité.

\item Point sur l’information en santé et sécurité dans l’entreprise.

\item Questions, initiatives et propositions des représentants du personnel…
\end{itemize}
\end{frame}


%%%%%%%%%%%%%%
%%% diapo 57 %%%
%%%%%%%%%%%%%%
\begin{frame}
\frametitle{Exemples « ordres du jour »}

\textbf{Comment vos ordres du jour sont ils structurés ?}
\end{frame}

%%%%%%%%%%%%%%
%%% diapo 58 %%%
%%%%%%%%%%%%%%
\begin{frame}
\frametitle{L’ordre du jour : élément essentiel et réglementé}

\textbf{Rôle du secrétaire :}

\begin{itemize}
\item établit l’ordre du jour avec le président,

\item représente le CHSCT au CE,

\item rédige également le PV avec le président,

\item conserve les archives du CHSCT,

\item pour chaque thème, il doit noter : les éléments essentiels décrivant le sujet traité, les avis exprimés, les décisions prises, la date de réalisation prévue, le responsable.
\end{itemize}
\end{frame}

%%%%%%%%%%%%%%
%%% diapo 59 %%%
%%%%%%%%%%%%%%
\begin{frame}
\frametitle{Le PV CHSCT}

\textbf{Le PV CHSCT}

\begin{itemize}
\item Obligatoire = mémoire du CHSCT.

\item Synthétique, clair, objectif et complet.

\item Peut être affiché sauf si information confidentielle.

\item Approuvé lors de la réunion suivante et joint à la convocation.

\item Reflet fidèle des réunions.

\item Préciser : nom & qualité des participants, présents, excusés, absents.

\item Conservés et tenus à la disposition de l’inspecteur du travail et des agents de la CARSAT.
\end{itemize}
\end{frame}

%%%%%%%%%%%%%%
%%% diapo 60 %%%
%%%%%%%%%%%%%%
\begin{frame}
\frametitle{Le PV CHSCT}

\textbf{Rôle du secrétaire}
\begin{itemize}
\item rédige le PV avec le président,

\item pour chaque thème, il doit noter : 
\begin{enumerate}
\item \textbf{les éléments essentiels décrivant le sujet traité,}

\item \textbf{les avis exprimés,} 

\item \textbf{les décisions prises,} 

\item \textbf{la date de réalisation prévue,} 

\item \textbf{le responsable.}
\end{enumerate}
\end{itemize}
\end{frame}

%%%%%%%%%%%%%%
%%% diapo 61 %%%
%%%%%%%%%%%%%%
\begin{frame}
\frametitle{Exemples PV}

\end{frame}

%%%%%%%%%%%%%%
%%% diapo 62 %%%
%%%%%%%%%%%%%%
\begin{frame}
\frametitle{Un CHSCT efficace = interventions sur le terrain}


\textbf{Principe de base :  La liberté de circulation des membres du CHSCT est entière dans et hors de l’établissement.}

\textit{Visites}

\begin{itemize}
\item La visite d’un atelier ou d’un service est une pratique courante dans le fonctionnement des CHSCT. L’inspection est plus rare…
\end{itemize}
\textbf{Inspections régulières, études et enquêtes}
\begin{itemize}
\item \textit{Etudes} : recherche de solutions relatives à l'organisation matérielle du travail, à l'environnement physique du travail, à l'aménagement des lieux de travail, des postes de travail, à la durée et aux horaires de travail. Le CHSCT devra également étudier l'incidence de l'introduction de toute technologie nouvelle sur les conditions de travail dans l'établissement. Au besoin, s'il l'estime nécessaire, il peut s'adresser à des experts agréés.
\item \textit{Enquêtes} : effectuées en cas d‘AT/MP, ou en cas d'incidents répétés ayant révélé un risque grave. 
\end{itemize}
\end{frame}

%%%%%%%%%%%%%%
%%% diapo 63 %%%
%%%%%%%%%%%%%%
\begin{frame}
\frametitle{Outil de visite : La préparation de la visite}

\textbf{schéma}
\end{frame}

%%%%%%%%%%%%%%
%%% diapo 64 %%%
%%%%%%%%%%%%%%
\begin{frame}
\frametitle{Outil de visite : La grille d’observation}

\textbf{schéma}
\end{frame}

%%%%%%%%%%%%%%
%%% diapo 65 %%%
%%%%%%%%%%%%%%
\begin{frame}
\frametitle{Analyse des conditions de travail}
\framesubtitle{1er outil : la visite de zone, santé/sécurité}
\textbf{schéma}
\end{frame}

%%%%%%%%%%%%%%
%%% diapo 66 %%%
%%%%%%%%%%%%%%
\begin{frame}
\frametitle{Analyse des conditions de travail}
\framesubtitle{Visite santé/sécurité}
\textbf{schéma}
\end{frame}


%%%%%%%%%%%%%%
%%% diapo 67 %%%
%%%%%%%%%%%%%%
\begin{frame}
\frametitle{Un CHSCT efficace = interventions sur le terrain}

\textbf{L’inspection}
\begin{itemize}
\item C’est l’inspecteur qui a l’initiative : il sait précisément ce qu’il vient contrôler :

\begin{enumerate}
\item le respect par l’employeur, des textes législatifs et réglementaires pris dans le domaine de l’hygiène, de la sécurité et des conditions de travail.
\end{enumerate}
\item L’inspecteur possède aussi la maîtrise du temps. 

\item Il peut toujours pousser son contrôle plus dans le détail ou décider que cela lui suffit et arrêter son inspection… 

\item Par rapport aux salariés dont les ateliers sont inspectés, le membre du CHSCT a alors la qualité d’un point d’appui. Le pouvoir de sa fonction est un contre-pouvoir, celui donné aux salariés par les institutions représentatives du personnel.
\end{itemize}
\end{frame}

%%%%%%%%%%%%%%
%%% diapo 68 %%%
%%%%%%%%%%%%%%
\begin{frame}
\frametitle{Un CHSCT efficace = interventions sur le terrain}

\textbf{L’inspection}

Par cette mission d’inspection :
\begin{itemize}
\item Les représentants du personnel au CHSCT sont de fait dans le même rôle de contrôle que l’inspecteur du travail. En effet, si le Code du Travail prévoit une mission d’inspection, c’est comme moyen pour que le CHSCT « veille à l’observation des prescriptions législatives et réglementaires prises en la matière ».

\item Les membres du CHSCT doivent se donner les moyens d’inspecter.
\begin{enumerate}
\item il faut posséder les textes applicables à ce que l’on veut inspecter.
\item il faut choisir à l’avance les points sur lesquels portera le contrôle.
\end{enumerate}
\item L’obtention des informations nécessaires sera donc un objectif stratégique préalable à l’inspection elle-même.
\end{itemize}
\end{frame}

%%%%%%%%%%%%%%
%%% diapo 69 %%%
%%%%%%%%%%%%%%
\begin{frame}
\frametitle{Un CHSCT efficace = interventions sur le terrain}

\textbf{Les missions d’enquêtes}

\begin{itemize}
\item Le CHSCT doit effectuer des enquêtes en matière d'AT/MP.

\item Effectuées, si possible, conjointement par au moins 2 représentants du personnel au CHSCT et le chef d'établissement ou son représentant.

\item \textbf{En cas d'enquête} (à la suite d'un AT grave, ou d'incidents répétés ayant révélé un rique grave, ou d'une MP ou MCP grave, ou en vue de rechercher des mesures préventives dans toute situation d'urgence et de gravité, la fiche de renseignement prévue par l'arrêté du 8 août 1986 sera \textbf{établie et adressée dans les délais (15 jours) à l'inspecteur du travail}.
\end{itemize}
\end{frame}

%%%%%%%%%%%%%%
%%% diapo70 %%%
%%%%%%%%%%%%%%
\begin{frame}
\frametitle{Un CHSCT efficace = interventions sur le terrain}

\textbf{schéma}
\end{frame}


%%%%%%%%%%%%%%
%%% diapo 71 %%%
%%%%%%%%%%%%%%
\begin{frame}
\frametitle{Le Délit d’entrave : contre le CHSCT}

Si, malgré les précautions prises, le fonctionnement du CHSCT est entravé, c'est le tribunal correctionnel qui est compétent (Article L.4742-1):

\textit{« Le fait de porter atteinte ou de tenter de porter atteinte soit à la constitution, soit à la libre désignation des membres, soit au fonctionnement régulier du comité d'hygiène, de sécurité et des conditions de travail, notamment par la méconnaissance des dispositions du livre IV de la deuxième partie relatives à la protection des représentants du personnel à ce comité, est puni d'un emprisonnement d'un an et d'une amende de 3 750 \euro{} ».}
\end{frame}

%%%%%%%%%%%%%%
%%% diapo 72 %%%
%%%%%%%%%%%%%%
\begin{frame}
\frametitle{Présentation de formulaires}

\begin{itemize}
\item Formulaire : \textbf{ENQUÊTE DU COMITÉ D'HYGIÈNE, DE SÉCURITÉ ET DES CONDITIONS DE TRAVAIL} \textit{RELATIVE À UN ACCIDENT DU TRAVAIL GRAVE}.

\item Formulaire : \textbf{ENQUÊTE DU COMITÉ D'HYGIÈNE, DE SÉCURITÉ ET DES CONDITIONS DE TRAVAIL RELATIVE} \textit{À UNE SITUATION DE TRAVAIL RELEVANT DE MALADIE PROFESSIONNEL GRAVE OU A CARACTERE PROFESSIONNEL}.

\item Formulaire : \textbf{ENQUÊTE DU COMITÉ D'HYGIÈNE, DE SÉCURITÉ ET DES CONDITIONS DE TRAVAIL RELATIVE}  \textit{A DES SITUATIONS DE RISQUE GRAVE…}.
\end{itemize}
\end{frame}

%%%%%%%%%%%%%%
%%% diapo 73 %%%
%%%%%%%%%%%%%%
\begin{frame}
\frametitle{Les documents administratifs}

\textbf{schéma}
\end{frame}

%%%%%%%%%%%%%%
%%% diapo 74 %%%
%%%%%%%%%%%%%%
\begin{frame}
\frametitle{Exemple de programme annuel de prévention des risques professionnels}

\textbf{schéma}
\end{frame}

%%%%%%%%%%%%%%
%%% diapo 75 %%%
%%%%%%%%%%%%%%
\begin{frame}
\frametitle{Synthèse}

\end{frame}

%%%%%%%%%%%%%%
%%% diapo 76 %%%
%%%%%%%%%%%%%%
\begin{frame}
\frametitle{Auto-diagnostic de fonctionnement}

\textbf{Plan d’action CHSCT}

\begin{itemize}
\item \textbf{Comment fonctionner (nombre de réunions...)}

\item \textbf{Organiser les visites}.

\item \textbf{Sensibiliser/informer le personnel (campagnes...)}.

\item \textbf{Se faire connaître en tant que CHSCT}.

\item \textbf{Analyser les incidents/accidents}.

\item \textbf{Comment être informé ou consulté...}.
\end{itemize}
\end{frame}

%%%%%%%%%%%%%%
%%% diapo 77 %%%
%%%%%%%%%%%%%%
\begin{frame}
\frametitle{Ce que peut faire le CHSCT}

\textbf{schéma}
\end{frame}

%%%%%%%%%%%%%%
%%% diapo 78 %%%
%%%%%%%%%%%%%%
\begin{frame}
\frametitle{Les ressources internes/externes}

\textbf{tableau}
\end{frame}

%%%%%%%%%%%%%%
%%% diapo 79 %%%
%%%%%%%%%%%%%%
\begin{frame}
\frametitle{Les 4 points clés de la « posture dynamique » pour faire face aux nouveaux risques et enjeux SST}

\textbf{schéma}
\end{frame}


%%%%%%%%%%%%%%
%%% diapo 80 %%%
%%%%%%%%%%%%%%
\begin{frame}
\frametitle{Conclusion}

\begin{itemize}
\item Constitué dans tous les établissements occupant au moins 50 salariés, le CHSCT a pour mission de contribuer à la protection de la santé et de la sécurité des salariés ainsi qu'à l'amélioration des conditions de travail. 

\item Composé notamment d'une délégation du personnel, le CHSCT dispose d'un certain nombre de moyens pour mener à bien sa mission (information, recours à un expert...  et les représentants du personnel, d'un crédit d'heures et d'une protection contre le licenciement. 

\item Ces moyens sont renforcés dans les entreprises à haut risque industriel. 

\item En l'absence de CHSCT, ce sont les délégués du personnel qui exercent les attributions normalement dévolues au comité. 

\item Le médecin du travail est invité à y assister avec une voix consultative.
\end{itemize}
\end{frame}
\end{document}




